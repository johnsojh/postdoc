\documentclass[12pt]{article}

\usepackage[margin=1in]{geometry}
\usepackage[doublespacing]{setspace}
\usepackage{amsthm, amssymb, amsmath}
% \usepackage[all]{xy}
\usepackage{booktabs}

\newtheoremstyle{plain}{3mm}{3mm}{\slshape}{}{\bfseries}{.}{.5em}{}
\theoremstyle{plain}
\newtheorem{thm}{Theorem}[section]
\newtheorem{vdw}[thm]{Van der Waerden's Theorem}
\newtheorem{fst}[thm]{Hindman's Theorem}
\newtheorem{schur}[thm]{Schur's Theorem}
\newtheorem{sz}[thm]{Szemer\'{e}di's Theorem}

\theoremstyle{definition}
\newtheorem{defn}[thm]{Definition}
\newtheorem{rmk}[thm]{Remark}
\newtheorem{ques}[thm]{Question}
\newtheorem{prob}[thm]{Problem}

\newcommand{\bbN}{\mathbb{N}}
\newcommand{\VDW}{\mathcal{VDW}}
\newcommand{\la}{\langle}
\newcommand{\ra}{\rangle}

\title{Research Statement}
\author{John H.~Johnson}

\begin{document}
\maketitle
\tableofcontents
\section{Research Summary}
I'm interested in Ramsey Theory, Topological Algebra, and the
interplay between these two fields.
In Ramsey Theory, there are many statements inspired by Hindman's
Theorem and van der Waerden's Theorem.
We can extend the usual addition on $\bbN$ to $\beta\bbN$ to give the
Stone-\v{C}ech compactification the structure of a compact Hausdorff
right-topological semigroup.
Using this structure, we have that Hindman-type theorems correspond to
idempotent elements of $\beta\bbN$, while van der Waerden-type
theorems correspond to compact ideals.
With this correspondence, combinatorial results produce interesting
algebraic consequences.
Conversely, knowledge about the algebra of $\beta\bbN$ often yields
relatively easy proofs of combinatorial results. 

I am also interested in mathematical outreach and increasing the
interest and participation of minority students in the mathematical
sciences. 
I was able to gain experience in both of these aspects during my two
year tenure as a GK-12 Fellow.
Since this experience was very valuable, for both me and the students,
I would like to setup such a program at $\la\hbox{place}\ra$.
In addition, I would also like to develop a partnership between Howard
University and $\la\hbox{place}\ra$ such that there is an exchange of
late stage graduate students. 
(Ideally the exchange would last an academic year.)
With this exchange students will be able to teach at their host
institution, write their dissertation, present their research, and
network. 

\section{Research Description}
\subsection{Hindman's Theorem}
Issai Schur, in \cite{Schur:1916fk}, proved one of the earliest
results in Ramsey Theory.
His theorem states that if the natural numbers are divided into
finitely many classes, then there exists natural numbers $x$ and $y$
such that $\{\,x, y, x+y\,\}$ is contained in one class.
Several decades later in \cite{Hindman:1974ys}, Neil Hindman proved a
strong generalization of Schur's Theorem.
(It's known that Hindman's Theorem is strictly stronger than Schur's
Theorem \cite[Theorems 16.28 and 16.29]{Hindman:1998fk}; moreover,
Hindman's Theorem can \textsl{not} be generalized to an uncountable
``index'' \cite{Milliken:1978fk}.)
\begin{fst}
  For all natural numbers $r$, if \/ $\bbN = \bigcup_{i=1}^r C_i$,
  then there exist $i \in \{1, 2, \ldots, r\}$ and a sequence $\la x_n
  \ra_{n=1}^\infty$ in $\bbN$ such that
  \[
    \biggl\{\, \sum_{n \in F} x_n : \hbox{$\emptyset \ne F \subseteq
      \bbN$ is finite} \,\biggr\} \subseteq C_i.
  \]
\end{fst}

The original proof of this theorem uses a complicated combinatorial
argument. 
(However, Baumgartner gives a shorter and slightly easier
combinatorial proof in \cite{Baumgartner:1974uq}.)
Even before the first combinatorial proof was discovered, it was known
by Fred Galvin that an easy proof of Hindman's Theorem would follow
from the existence of an object called an idempotent ultrafilter.
Unfortunately lack of space prohibits us from expanding on the
interesting history of Hindman's Theorem; but, the interested reader
can find a brief summary in the notes section of \cite[Chapter
5]{Hindman:1998fk}.

This connection between Hindman's Theorem and idempotent ultrafilters
was one of the first of many interesting connections between Ramsey
Theory and ultrafilters.
Moreover, despite over thirty years of study there are still many
attractive open questions surrounding this connection. 
Before we can even state those particular questions that interest us,
we pause to give a brief outline of some basic results and
definitions. 

\subsection{Algebra in the Stone-\v{C}ech Compactification}
\begin{defn}
  \label{defn:uf}
  Let $\emptyset \ne p \subseteq \mathcal{P}(\bbN)$.
  We call $p$ an \textsl{ultrafilter on $\bbN$} if and only if it
  satisfies the following four properties:
  \begin{itemize}
    \item[(1)] $\emptyset \not\in p$.
    \item[(2)] If $A$, $B \in p$, then $A \cap B \in p$.
    \item[(3)] If $A \in p$ and $A \subseteq B \subseteq \bbN$,
      then $B \in p$.
    \item[(4)] Let $r \in \bbN$ and for each $i \in \{1, 2,
      \ldots, r\}$ suppose we have $C_i \subseteq \bbN$. 
      If $\bigcup_{i=1}^r C_i \in p$, then there exists $i \in
      \{1, 2, \ldots, r\}$ such that $C_i \in p$.
  \end{itemize}
\end{defn}

It's easy to construct essentially trivial ultrafilters. 
For all $a \in \bbN$ put $e(a) = \{\, A \subseteq \bbN : a \in A
\,\}$.
Then a simple check shows that $e(a)$ is an ultrafilter on $\bbN$.
We call such ultrafilters \textsl{principal ultrafilters}. 
Ultrafilters that are not principal are called \textsl{nonprincipal
  ultrafilters}, and the existence of such ultrafilters follows from
Zorn's Lemma. 

Our next set of definitions will enable us to describe a topological
and algebraic structure on the collection of all ultrafilters on $\bbN$.
This will give the Stone-\v{C}ech compactification of $\bbN$ the
structure of a compact Hausdorff right-topological semigroup. 
(We will explain what ``right-topological'' means shortly.)
\begin{defn}
  \label{defn:alg}
  \begin{itemize}
    \item[(a)] $\beta\bbN = \{\, p \subseteq \mathcal{P}(\bbN) :
      \hbox{$p$ is an ultrafilter on
      $\bbN$} \,\}$.
    \item[(b)] For all $A \subseteq \bbN$, put $\overline{A} = \{\,
      p \in \beta\bbN : A \in p \,\}$.
    \item[(c)] For all $A \subseteq \bbN$ and $x \in \bbN$, put
      $-x+A = \{\, y \in \bbN : x+y \in A \,\}$.
    \item[(d)] For $p$, $q \in \beta\bbN$, put
      $p+q = \bigl\{\, A \subseteq \bbN : \{\, x \in \bbN : -x +A \in q
      \,\} \in p \,\bigr\}$.
    \item[(e)] We say $p \in \beta\bbN$ is an \textsl{idempotent
        (ultrafilter)} if  and only if $p + p = p$.
  \end{itemize}
\end{defn}

The collection $\{\, \overline{A} : A \subseteq \bbN \,\}$ is a
basis for a compact Hausdorff topology on $\beta\bbN$.
We identify $\bbN$ with the principal ultrafilters by the injective
function $a \mapsto e(a)$. 
Then with this topology $\beta\bbN$ is simply the Stone-\v{C}ech
compactification of $\bbN$. 
For all $p$, $q \in \beta\bbN$, the sum $p+q$ is an ultrafilter.
Moreover, $(\beta\bbN, +)$ is a semigroup where the addition on
$\beta\bbN$ is an extension of the usual addition on $\bbN$. 
The details behind these algebraic assertions can be found in
\cite[Chapter 4 Section 1]{Hindman:1998fk}.

We summarize our discussion by saying that $(\beta\bbN,
+)$ is a compact Hausdorff right-topological semigroup. 
The right-topological semigroup part means that for all $p \in
\beta\bbN$, the function $q \mapsto q+p$ is continuous.
It's a fact due to Robert Ellis \cite[Corollary 2.10]{Ellis:1969zr}
that every compact Hausdorff right-topological semigroup contains an
idempotent. 
In the case of $\beta\bbN$, we have 
$2^{2^{\aleph_0}}$ idempotents \cite[Theorem 6.44]{Hindman:1998fk}. 

To demonstrate how all of this elaborate equipment can be put to use,
we give an easy proof of Schur's Theorem using an idempotent.
\begin{schur}
  For every natural number $r$, if  \/ $\bbN = \bigcup_{i=1}^r C_i$,
  then there exist $i \in \{1, 2, \ldots, r\}$ and natural numbers
  $x$ and $y$ such that $\{\, x, y, x+y \,\} \subseteq C_i$.
\end{schur}
\begin{proof}
  Pick an idempotent $p$ in $\beta\bbN$.
  Since $\bigcup_{i=1}^r C_i = \bbN \in p$, by Definition
  \ref{defn:uf}(4) pick $i \in \{1, 2, \ldots, r\}$ such that $C_i \in
  p$.
  Now by Definition \ref{defn:alg}(d) we have that $\{\, x \in \bbN :
  -x + C_i \in p \,\} \in p$; and, Definition \ref{defn:uf}(2) shows
  that $C_i \cap \{\, x \in \bbN : -x + C_i \in p \,\} \in p$.
  Hence we can pick $x \in C_i$ such that $-x + C_i \in p$. 
  Applying Definition \ref{defn:uf}(2) again yields $C_i \cap (-x +
  C_i) \in p$.
  Finally, we can pick $y \in C_i \cap (-x+C_i)$.
  Therefore we have $\{\, x, y, x+y \,\} \subseteq C_i$.
\end{proof}

It's possible to strengthen the conclusion of this theorem in several
ways.
One strengthen is to show that the set $\bigl\{\,\{\, x, y \,\}
\subseteq \bbN : \{\,x,y,x+y\,\} \subseteq C_i \,\bigr\}$ is infinite.
Another strengthen is to show that there exists a sequence $\la x_n
\ra_{n=1}^\infty$ in $C_i$ such that $\{\, x_m + x_n : m \ne n \,\}
\subseteq C_i$.
(Both of these results can be derived easily by knowing that members
of nonprincipal ultrafilters are infinite, and idempotents in
$\beta\bbN$ are necessarily nonprincipal.)
Modifying the proof to produce Hindman's Theorem is mainly a matter of
formulating the right inductive hypothesis.
The interested reader can find the details in \cite[Theorem
5.8]{Hindman:1998fk}. 

Our final set of basic definitions we allow us to formulate another type
of important algebraic structure in $(\beta\bbN, +)$. 
\begin{defn}
  Given subsets $A$ and $B$ of $\beta\bbN$, we put $A + B = \{\, p+q :
  \hbox{$p \in A$ and $q \in B$} \,\}$.
  Let $\emptyset \ne J \subseteq \beta\bbN$.
  \begin{itemize}
    \item[(a)] $J$ is a \textsl{left ideal} if and only if
      $\beta\bbN + J \subseteq J$.
    \item[(b)] $J$ is a \textsl{right ideal} if and only if
      $J+\beta\bbN \subseteq J$.
    \item[(c)] $J$ is a \textsl{(two-sided) ideal} if and only if
      $J$ is both a left and right ideal.
    \item[(d)] $\bbN^* = \{\, p \in \beta\bbN : \hbox{$p$ is
        nonprincipal} \,\}$.
    \item[(e)] We let $K(\beta\bbN)$ represent the smallest ideal of
      $(\beta\bbN, +)$, that is, $K(\beta\bbN)$ is a two-sided ideal
      and it is contained in every ideal of $(\beta\bbN, +)$.
  \end{itemize}
\end{defn}

By \cite[Theorem 4.36]{Hindman:1998fk}, it's known that $\bbN^*$ is a
compact ideal; and by \cite[Theorem 2.8]{Hindman:1998fk}, it's known
that the smallest ideal $K(\beta\bbN)$ exists. 
(However, $K(\beta\bbN)$ is \textsl{not} closed. 
This fact follows from \cite[Theorem 4.4]{Hindman:1996uq}.)
The smallest ideal is one of the most well understood and studied
ideal in $(\beta\bbN, +)$.
For instance, elements in $K(\beta\bbN)$ are, in several ways,
``close'' to being idempotents ultrafilters. 
(See for example \cite[Theorems 4.39 and 4.43]{Hindman:1998fk}.)
While the smallest ideal has a central importance in $(\beta\bbN, +)$,
it is also the case that $c\ell\bigl(K(\beta\bbN)\bigr)$ is an
important compact ideal of $\beta\bbN$. 
For instance, $c\ell\bigl(K(\beta\bbN)\bigr)$ is contained in every
compact ideal of $(\beta\bbN, +)$, and many theorems true for
$K(\beta\bbN)$ are also true for $c\ell\bigl(K(\beta\bbN)\bigr)$ (with
a ``few'' modifications). 
Hence, many of our questions and problems are motivated by results
about $c\ell\bigl(K(\beta\bbN)\bigr)$.

\subsection{Open Questions and Problems}
Let $\Delta_1$ denote $c\ell\bigl(K(\beta\bbN)\bigr)$.
Since $\bbN^*$ is an ideal, a simple check proves that $\Delta_1 +
\bbN^*$ is a right ideal.
From \cite[Theorem 2.19(a)]{Hindman:1998fk}, we have that
$c\ell(\Delta_1 + \bbN^*)$ is a compact ideal.
An easy argument, using the fact that $\Delta_1$ is the smallest
compact ideal, shows that $\Delta_1 = c\ell(\Delta_1+\bbN^*)$.
It is natural to ask whether $\Delta_1$ is the only compact
ideal with this ``fixed-point'' property.
More precisely, we ask the following question.
\begin{ques}
  Does there exists a compact ideal $J$ such that $J \ne \Delta_1$ and
  $J = c\ell(J+\bbN^*)$?
\end{ques}
In \cite[Lemma 3.3]{Davenport:1987uq}, Davenport and Hindman were able
to show the more precise result that $\Delta_1 = c\ell(p+\bbN^*)$ if
and only if $p \in \Delta_1$.
This particular result naturally leads to the following question.
\begin{ques}
  Does there exists a compact ideal $J$ such that $J \ne \Delta_1$ and
  $J = c\ell(p + \bbN^*)$ if and only if $p \in J$?
\end{ques}

Davenport and Hindman called $c\ell(p+\bbN^*)$, for $p \in \beta\bbN$,
a \textsl{subprincipal (closed) ideal}.
To state one of the open questions in \cite{Davenport:1987uq} relating
to subprincipal ideals we make the following list of definitions.
\begin{defn}
  Let $A \subseteq \bbN$.
  \begin{itemize}
    \item[(a)] The \textsl{upper density} of $A$ is 
      \[
        \overline{d}(A) = \limsup_{N\to\infty} \frac{|A \cap \{1, 2,
          \ldots, N\}|}{N}.
      \]
    \item[(b)] The \textsl{Banach density} of $A$ is 
      \begin{align*}
        d^*(A) &= \sup\bigl\{\, \alpha \in [0,1] : \hbox{for all $k \in
          \bbN$, there exist $n \ge k$ and $a \in \bbN$} \\
        &\hspace{5em}\hbox{such that $|A \cap \{\,a, a+1, \ldots,
          a+n-1\,\}| \ge \alpha\cdot n$} \,\bigr\}.
      \end{align*}

    \item[(c)] $\Delta = \{\, p \in \beta\bbN : \hbox{$\overline{d}(A)
        > 0$ for all $A \in p$} \,\}$.

    \item[(d)] $\Delta^* = \{\, p \in \beta\bbN : \hbox{$d^*(A) > 0$
        for all $A \in p$} \,\}$.
  \end{itemize}
\end{defn}
It's an exercise to see that $\overline{d}(A) \le d^*(A)$.
(Moreover, $A = \bigcup_{n=1}^\infty \{\,2^n, 2^n +1, \ldots, 2^n+n
\,\}$ shows that this inequality can become a strict inequality.)
It's known that $\Delta$ is a compact left ideal \cite[Theorem
6.79]{Hindman:1998fk} and $\Delta^*$ is a compact ideal \cite[Theorem
20.5]{Hindman:1998fk}.
By our inequality $\overline{d}(A) \le d^*(A)$, it follows that
$\Delta \subseteq \Delta^*$.
We now present the open question in \cite{Davenport:1987uq}.
\begin{ques}
  Is $\Delta^*$ a subprincipal ideal?
\end{ques}

By way of contrast to this question, in \cite[Lemma
3.4]{Hindman:1988kx}, Hindman proved that $\Delta^* = c\ell(\Delta +
\bbN^*)$.  
Therefore this particular result leads us to our next question.
\begin{ques}
  Does there exist a compact ideal $J$ with a compact left ideal $L
  \subsetneq J$ such that $J \ne \Delta^*$, $J \ne \Delta_1$, and $J =
  c\ell(L + \bbN^*)$?
\end{ques}
\begin{rmk}
  We have to exclude the ideal $\Delta_1$ too, since it is known that
  we can pick a compact left ideal $L \subsetneq \Delta_1$ such that
  $\Delta_1 = c\ell(L + \bbN^*)$.
\end{rmk}

Bartel Leendert van der Waerden, in \cite{Van-der-Waerden:1927fk},
proved that if the natural numbers are divided into finitely many
classes, then for every natural number $k$ at least one class contains
a $k$-term arithmetic progression.
Van der Waerden's proof was purely elementary, using only the
pigeonhole principle and induction. 
However, an easy proof using the algebraic structure of $\Delta_1$ is
also known (see \cite[Theorem 14.1 and Corollary 14.2]{Hindman:1998fk}).
Many decades later, Szemer\'{e}di proved a powerful generalization of van
der Waerden's Theorem in \cite{Szemeredi:1975uq}.
Szemer\'{e}di proved that if $d^*(A) > 0$ and $k$ is a natural number,
then $A$ contains a $k$-term arithmetic progression.

While Szemer\'{e}di's Theorem has many different proofs, there is
currently no known proof that uses the algebraic structure of
$(\beta\bbN,+)$. 
Of course it is not obvious that such a proof, even in principle,
exists.
Moreover, it is difficult to imagine how such a proof could exist
without a more detailed understanding of the algebraic structure of
compact ideals. 
Accordingly, the above questions are meant to improve this understanding.
But, even if all of these questions can be answered it is likely that
an algebraic proof of Szemer\'{e}di's Theorem is likely to still be a
very difficult problem. 

However, a couple of decades before Szemer\'{e}di proved his famous
theorem, Roth in \cite{Roth:1953fk} proved that if $d^*(A) > 0$, then
$A$ contains a 3-term arithmetic progression. 
The proof of Roth's Theorem is known to be (by certain technical
considerations) simpler to prove than Szemer\'{e}di's Theorem. 
I'm doubtful that answers to the above questions will lead to an
algebraic proof of Szemer\'{e}di's Theorem; but, I am more optimistic that
answers to the above questions \textsl{can} lead to an algebraic proof
of Roth's Theorem.
Therefore we propose the following (hopefully simpler) problem.
\begin{prob}
  Produce an algebraic proof of Roth's Theorem.
  By algebraic, we roughly mean a proof that mainly uses the
  algebraic structure of $(\beta\bbN, +)$. 
\end{prob}
\begin{rmk}
  One possible idea to prove Szemer\'{e}di's Theorem algebraically is
  contained in \cite{Davenport:1987uq}.  
  In \cite[Theorem 3.8]{Hindman:1982zr}, Hindman proved that 
  \begin{align*}
    \Delta_1 &= \{\, p \in \beta\bbN: \hbox{for all $A \in p$, there
      exists $k \in \bbN$} \\
      &\hspace{5em}\hbox{such that $d^*\bigl(\bigcup_{i=1}^k
        -i+A\bigr) = 1$} \,\}.
  \end{align*}
  Also, in \cite[Theorem 3.8]{Hindman:1982fk}, he proved that for
  every $\varepsilon >0$, if $d^*(A) > 0$, then there exists a natural
  number $k$ such that $d^*\bigl(\bigcup_{i=1}^k -i+A \bigr) \ge 1 -
  \varepsilon$. 
  Hence, in a certain sense, $\Delta^*$ is ``close'' to the ideal
  $\Delta_1$. 
  It was conjectured by Davenport and Hindman that this ``closeness''
  would imply that $\Delta^*$ is contained in every compact ideal of
  $(\beta\bbN, +)$. 
  If this conjecture were true, it would imply Szemer\'{e}di's
  Theorem. 
  Unfortunately, this promising conjecture turned out to be
  false. 
  The paper \cite{Davenport:1987uq} produces several compact ideals
  that do not contain $\Delta^*$. 
  However another interesting result contained in this paper is that
  $\Delta^*\setminus\Delta_1$ intersects every compact ideal of
  $(\beta\bbN, +)$.

  It may be possible to modify this result to prove Roth's Theorem. 
  It's known that $\bbN^* = \{\, p \in \beta\bbN : \hbox{$A$ contains
    a 2-term arithmetic progression for all $A \in p$} \,\}$ and that
  $\mathcal{AP}_3 = \{\, p \in \beta\bbN : \hbox{$A$ contains a }
  \hbox{3-term arithmetic progression for all $A \in p$} \,\}$ is a
  compact ideal.
  Hence if $\mathcal{AP}_3$ is ``close'' enough to $\bbN^*$, it may be
  possible to prove, since $\Delta^* = c\ell(\Delta+\bbN^*)$ that
  $\Delta^* \subseteq  \mathcal{AP}_3$. 
  This last inclusion is equivalent to Roth's Theorem.
\end{rmk}



To state our next and final question, we make the following definition. 
\begin{defn}
  \begin{itemize}
    \item[(a)] Let $A \subseteq \bbN$. 
      We say $A$ is a $J$-set if and only if for every nonempty finite
      collection $F$ of sequences in $\bbN$, there exist $a \in \bbN$
      and a finite nonempty subset $H \subseteq \bbN$ such that for
      all $f \in F$,
      \[
        a + \sum_{t \in H} f(t) \in A.
      \]
    \item[(b)] $J(\bbN) = \{\, p \in \beta\bbN : \hbox{$A$ is a
        $J$-set for all $A \in p$} \,\}$.
  \end{itemize}
\end{defn}

It's known that if $d^*(A) > 0$, then $A$ is a $J$-set \cite[Theorem
6.10]{Hindman:2009vn}.
Hence $\Delta^* \subseteq J(\bbN)$.   
It's also known that there exists a $J$-set $B$
with $d^*(B) = 0$ \cite[Theorem 2.1]{Hindman:2009ys}.
Since we know that there exists $J$-sets with zero Banach density, our
next question asks if there is a notion of density such that $J$-sets
have positive density.
\begin{ques}
  Does there exists a function $\delta \colon \mathcal{P}(\bbN) \to
  [0, \infty)$ with the following properties? 
  \begin{itemize}
    \item[(1)] For all $A \subseteq \bbN$, $\delta(A) \ge d^*(A)$.
    \item[(2)] For all $t \in \bbN$, $\delta(A) = \delta(-t+A)$.
    \item[(3)] If $A \subseteq \bbN$ is a $J$-set, then $\delta(A) >
      0$.
    \item[(4)] If $\delta(A \cup B) > 0$, then either $\delta(A) >0$
      or $\delta(B) > 0$.
    \item[(5)] The set
      \[
        \{\, p \in \beta\bbN: \hbox{for all $A \in p$, there
          exists $k \in \bbN$} 
        \hbox{ such that $\delta\bigl(\bigcup_{i=1}^k
          -i+A\bigr) = 1$} \,\}.
      \]
      is a compact ideal. 
  \end{itemize}
\end{ques}
\begin{rmk}
  We require property (5) since we are trying to draw certain
  analogies between $\Delta^*$ and $J(\bbN)$. 
\end{rmk}

\section{Broader Impact}
\subsection{GK-12 Program}
For two years at Howard University I was a GK-12 Graduate Student
Fellow. 
The purpose of the GK-12 program was to increase 
minority students' interest in science, technology, engineering, and
mathematics by pairing a class of students with a graduate
student. 
When I participated in the program, I was paired with a high school
teacher and two of his classes. 
Each week I would prepare a mathematical activity for the students and
go into the classroom to run the activity.

Overall the students enjoyed the activities, and I learned quite a bit
about how to communicate difficult mathematical ideas to an audience
with varying abilities and interest. 
Moreover, I also think that the activities help to expand the
students' view of the great variety and applicability of mathematics. 

I propose to continue this program at $\la\hbox{place}\ra$.
My first year, I will make contact with a local middle or high school
so I can
come in weekly to a particular class for the school year and run
various mathematical activities. 
For the second and third year, I will apply for a grant so that a
couple of graduate students can also participate by developing
activities and running them in the classroom. 
In addition, we will also setup a website, and if possible, record new
or innoviate activities for wider dissemination. 

\subsection{Preparing Future Faculty}
Currently, I'm participating in the Preparing Future Faculty
internship program at James Madison University. 
This particular program is a partnership between Howard University and
James Madison University.
With this program
graduate students that are in the late stages of their research can go
to a host institution  for a year to gain teaching experience, 
write up their dissertation, present their research, and gain new
contacts.

Since this is a valuable experience, 
I propose to create a partnership between Howard University and
$\la\hbox{place}\ra$ so that the two instituions can both gain from
the Preparing Future Faculty program.


\bibliographystyle{amsplain}
\addcontentsline{toc}{section}{References}
\bibliography{references}
\end{document}

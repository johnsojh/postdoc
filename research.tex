\documentclass[12pt]{article}

\usepackage[margin=1in]{geometry}
\usepackage[doublespacing]{setspace}
\usepackage{amsthm, amssymb, amsmath}
\usepackage[all]{xy}

\newtheoremstyle{plain}{3mm}{3mm}{\slshape}{}{\bfseries}{.}{.5em}{}
\theoremstyle{plain}
\newtheorem{thm}{Theorem}[section]
\newtheorem{vdw}[thm]{Van der Waerden's Theorem}
\newtheorem{fst}[thm]{Hindman's Theorem}

\theoremstyle{definition}
\newtheorem{defn}[thm]{Definition}

\newcommand{\bbN}{\mathbb{N}}
\newcommand{\VDW}{\mathcal{VDW}}
\newcommand{\la}{\langle}
\newcommand{\ra}{\rangle}


\begin{document}

\subsection{Van der Waerden's Theorem and Hindman's Theorem}
Van der Waerden's Theorem is one of the early important theorems in
Ramsey Theory \cite{Van-der-Waerden:1927fk} that asserts that if we
finitely partition the natural numbers, we can find a finite
arithmetic progression of any length in at least one class of the
partition.
More precisely we have
  \begin{vdw}
    For every natural number $r$, if\/ $\bbN = \bigcup_{i=1}^r C_i$,
    then for every natural number $k$ there exist $i \in \{1, 2,
    \ldots, r\}$ and natural numbers $a$ and $d$ such that $\{\, a,
    a+d, \ldots, a+(k-1)d\,\} \subseteq C_i$.
  \end{vdw}
Another important theorem in Ramsey Theory that was published
forty-seventy years after van der Waerden's result is Hindman's
Theorem.
  \begin{fst}
    For every natural number $r$, if\/ $\bbN = \bigcup_{i=1}^r C_i$,
    then there exist $i \in \{1, 2, \ldots, r\}$ and a sequence $\la x_n
    \ra_{n=1}^\infty$ of natural numbers such that $\{\, \sum_{n \in
      F} x_n : \emptyset \ne F \subseteq \bbN \hbox{ is finite} \,\}
    \subseteq C_i$.
  \end{fst}
Hindman's Theorem has a different character then van der Waerden's Theorem.

\section{Research Description}
\subsection{Van der Waerden's Theorem}
The field of Ramsey Theory is broadly described as the study of large
mathematical structures that are, in some sense, preserved under
finite partitions.
From the quantitative viewpoint, Ramsey Theorists investigate
estimates on how ``large'' a particular structure must be to induce
this preservation property.
(Except in a few small cases computing exactly how large a particular
structure must be is beyond current mathematical technology.)
Qualitatively however, the focus is on the properties of structures that are preserved, and how these structures are preserved, under finite partitions.
These two synergistic strands form the guiding principle for most
research in Ramsey Theory.
But, for this research statement we will almost exclusively focus on
the qualitative aspects of the field, and hence for us ``large'' will
almost always simply mean infinite.


Van der Waerden's Theorem is an early
result in Ramsey Theory that directly illustrates both these
quantitative and qualitative aspects.
Like most important theorems, there are many equivalent ways to state
van der Waerden's Theorem.
Since our focus is on qualitative results, we choose to give a
formulation that emphases the theorem's qualitative character. 
  \begin{vdw}
    
  \end{vdw}
In other words, if we finitely partition the natural numbers, we can
find a finite arithmetic progression of any length in at least one
class of the partition. 
Van der Waerden's Theorem naturally motivates several
qualitative questions. 
To help state our questions we introduce
the following nonstandard but suggestive notation.
  \begin{defn}
    For $A \subseteq \bbN$ and natural numbers $r$ and $k$, let
    $\VDW(A;r,k)$ represent the following statement
      \begin{quote}
        If $A = \bigcup_{i=1}^r C_i$, then there exist $i \in \{1, 2,
        \ldots, r\}$ and natural numbers $a$ and $d$ such that $\{\, a,
        a+d, \ldots, a+(k-1)d\,\} \subseteq C_i$.
     \end{quote}
  \end{defn}
Using this notation van der Waerden's Theorem asserts that for all
natural numbers $r$ and $k$ the statement $\VDW(\bbN;r,k)$ is true.
However, we can be a bit more precise than this!
A moments thought proves that for $A \subseteq \bbN$ and any natural
numbers $r$ and $k$ the following table of implications is valid.
  % Would be nice if I label this table.
  \begin{quote}
    \xymatrix{
      {\VDW(A;r,k)} \ar@{=>}[r]\ar@{=>}[d] & {\VDW(A;r-1,k)}
        \ar@{=>}[r]\ar@{=>}[d] & \cdots \ar@{=>}[r]\ar@{=>}[d] &
        {\VDW(A;1,k)} \ar@{=>}[d] \\
      {\VDW(A;r,k-1)} \ar@{=>}[r]\ar@{=>}[d] & {\VDW(A;r-1,k-1)}
        \ar@{=>}[r]\ar@{=>}[d] & \cdots \ar@{=>}[r]\ar@{=>}[d] &
        {\VDW(A;1,k-1)} \ar@{=>}[d] \\
      \vdots \ar@{=>}[d] & \vdots \ar@{=>}[d] 
        & \vdots \ar@{=>}[d]  & \vdots \ar@{=>}[d] \\
      {\VDW(A;r,1)} \ar@{=>}[r]& {\VDW(A;r-1,1)}\ar@{=>}[r] & 
        \cdots \ar@{=>}[r]& {\VDW(A;1,1)} \\
    }
  \end{quote}
Even though this table was figured out just purely off of the
definition of $\VDW(A; r,k)$, it does give us a hint at how the van
der Waerden's Theorem is proved.
Most proofs of van der Waerden's Theorem comes from the observation
that it suffices to prove for all natural numbers $r$ and $k$ the
following table of implications (reversed from what we had before).
  % Would be nice if I label this table.
  \begin{quote}
    \xymatrix{
      {\VDW(\bbN;r,k)} & {\VDW(\bbN;r-1,k)}
        \ar@{=>}[l] & \cdots \ar@{=>}[l] &
        {\VDW(\bbN;1,k)}\ar@{=>}[l]  \\
      {\VDW(\bbN;r,k-1)} \ar@{=>}[u] & {\VDW(\bbN;r-1,k-1)}
        \ar@{=>}[l]\ar@{=>}[u] & \cdots \ar@{=>}[l]\ar@{=>}[u] &
        {\VDW(\bbN;1,k-1)} \ar@{=>}[u]\ar@{=>}[l] \\
      \vdots \ar@{=>}[u] & \vdots \ar@{=>}[u] 
        & \vdots \ar@{=>}[u]  & \vdots \ar@{=>}[u] \\
      {\VDW(\bbN;r,1)} \ar@{=>}[u]& {\VDW(\bbN;r-1,1)}\ar@{=>}[l]
        \ar@{=>}[u] & 
        \cdots \ar@{=>}[l]\ar@{=>}[u]& {\VDW(\bbN;1,1)}\ar@{=>}[u]
        \ar@{=>}[l] \\
    }
  \end{quote}
Typically most proofs of van der Waerden's Theorem follows from
assuming that one row of the above table is true and proceeds to show
that the first entry in the next row up is ???

Using this notion we can vaguely ask what are the properties of
subsets $A \subseteq \bbN$ for which $\VDW(A)$ is true?

\bibliographystyle{plain}
\bibliography{references}
\end{document}

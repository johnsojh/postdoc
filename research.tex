% Draft of my Research Proposal for the Simons Postdoctoral Fellowship
% at Harvard University
% by: John H. Johnson
% email: john.j.jr@gmail.com

\documentclass[12pt]{article}

\usepackage{amsthm, amssymb, amsmath}

\newtheoremstyle{plain}{3mm}{3mm}{\slshape}{}{\bfseries}{.}{.5em}{}
\theoremstyle{plain}
\newtheorem*{vdw}{Van Der Waerden's Theorem}
\newtheorem*{roth}{Roth's Theorem}
\newtheorem*{sz}{Szemer\'{e}di's Theorem}
\newtheorem*{cst}{Central Sets Theorem}
\newtheorem*{thm}{Theorem}
\newtheorem*{cor}{Corollary}
\newtheorem*{prop}{Proposition}
\newtheorem*{lem}{Lemma}
\newtheorem*{ques}{Question}
\newtheorem*{conj}{Conjecture}
\newtheorem*{fact}{Fact}

\theoremstyle{definition}
\newtheorem*{defn}{Definition}
\newtheorem*{example}{Example}
\newtheorem*{rmk}{Remark}

\newcommand{\la}{\langle}
\newcommand{\ra}{\rangle}
\newcommand{\bbN}{\mathbb{N}}
\newcommand{\setfunc}[2]{\hbox{${}^{\hbox{$#1$}}\hskip -3 pt #2$}}

\begin{document}
\section{Research Summary}

\section{Research Description}
The field of Ramsey Theory can broadly be described as the study of
large mathematical structures that are preserved under finite
partitions. 
The qualitative aspect of Ramsey Theory focuses on finding precisely
how structures are preserved under finite partitions, and which
structures can be preserved. 
On the other hand, the quantitative aspect of Ramsey Theory focuses on
finding out exactly how ``large'' structures must be to have this
preservation property. 
(Similar to most other fields of mathematics, finding out exactly how
large a structure must be is beyond current mathematical technology. 
Instead we often settle for finding estimates.)

An early theorem in Ramsey Theory that nicely illustrates these
qualitative and quantitative aspects is van der Waerden's Theorem.

  \begin{vdw}
    For all $k$, $l \in \bbN$, there exists a number $W \in \bbN$ such
    that if $\{1, 2, \ldots, W \} = \bigcup_{i=1}^k C_i$, then there
    exist $i \in \{1, 2, \ldots, k\}$ and $a, d \in \bbN$ such that
    \[
      \{\, a, a+d, \ldots, a+(l-1)d \,\} \subseteq C_i.
    \]
  \end{vdw}

In order words, given a sufficiently large initial segment of the
natural numbers one of the cells of a partition of this segment must
contain a $l$ term arithmetic progression. 
Here the structures that are preserved is finite arithmetic
progressions, while the largeness is indicated by the number $W$.

  \begin{defn}
    For $k$ and $l \in \bbN$,
    the number $W(k,l) \in \bbN$ is the \textsl{van der Waerden
      number} if and only if it is the smallest number such that whenever
    $\{1, 2, \ldots, W(k,l)\} = \bigcup_{i=1}^k C_i$, then there
    exist $i \in \{1, 2, \ldots, k\}$ and $a$, $d \in \bbN$ with 
    \[
      \{\, a, a+d, \ldots, a+(l-1)d \,\} \subseteq C_i.
    \]
  \end{defn}

  \begin{defn}
    Let $l \in \bbN$. 
    For $N \in \bbN$ define 
      \begin{align*}
        r_l(N) &= \max\bigr\{\, A \subseteq \{1, 2, \ldots, N\} : 
          \hbox{there does not exist $a$,$b \in \bbN$ such that } \\
          &\hspace{5em}\hbox{$\{\,a, a+d, \ldots, a+(l-1)d \,\}
            \subseteq A$} \bigl\}.
      \end{align*}
  \end{defn}

  \begin{roth}
    $r_3(N) = O\bigl(\frac{N}{\log\log N}\bigl)$.
  \end{roth}
\section{Broader Impact}

\bibliographystyle{plain}
\bibliography{references}
\end{document}

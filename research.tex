% Draft of my Research Proposal for the Simons Postdoctoral Fellowship
% at Harvard University
% by: John H. Johnson
% email: john.j.jr@gmail.com

\documentclass[12pt]{article}

\usepackage{amsthm, amssymb, amsmath}

\newtheoremstyle{plain}{3mm}{3mm}{\slshape}{}{\bfseries}{.}{.5em}{}
\theoremstyle{plain}
\newtheorem*{vdw}{Van Der Waerden's Theorem}
\newtheorem*{roth}{Roth's Theorem}
\newtheorem*{sz}{Szemer\'{e}di's Theorem}
\newtheorem*{cst}{Central Sets Theorem}
\newtheorem*{thm}{Theorem}
\newtheorem*{cor}{Corollary}
\newtheorem*{prop}{Proposition}
\newtheorem*{lem}{Lemma}
\newtheorem*{ques}{Question}
\newtheorem*{conj}{Conjecture}
\newtheorem*{fact}{Fact}

\theoremstyle{definition}
\newtheorem*{defn}{Definition}
\newtheorem*{example}{Example}
\newtheorem*{rmk}{Remark}

\newcommand{\la}{\langle}
\newcommand{\ra}{\rangle}
\newcommand{\bbN}{\mathbb{N}}
\newcommand{\setfunc}[2]{\hbox{${}^{\hbox{$#1$}}\hskip -3 pt #2$}}

\begin{document}
\section{Research Summary}

\section{Research Description}
The field of Ramsey Theory is broadly described as the study of large
mathematical structures that are preserved under finite partitions.
From the qualitative viewpoint, we investigate precisely which
structures are preserved, and how structures are perserved, under
finite partitions.
Quantitatively however, we compute estimates on how ``large'' a
certain structure must be to achieve this preservation property. 
(Computing, except in a few small cases, exactly how large a structure
must be is beyond current mathematical technology.)
% This sentence may be a bit too bold!
These two synergistic strands is the guiding principle of all research
in Ramsey Theory.

Like most important theorems there are many equivalent ways to state
van der Waerden's Theorem.
We choose to give a formulation that is not traditional, but it has
the virtue of making the qualitative and quantitative aspects more
transparent. 

  \begin{vdw}
    For all $k$, $l \in \bbN$, there exists a natural number $W \in
    \bbN$ such that if $A \subseteq \bbN$ is any subset which contains a
    $W$-term arithmetic progression and $A = \bigcup_{i=1}^k C_i$,
    then there exist $i \in \{1, 2, \ldots, k\}$ and $a$, $d \in \bbN$
    such that 
      \[
        \{\, a, a+d, \ldots, a+(l-1)d \,\} \subseteq C_i.
      \]
  \end{vdw}

In order words, given any partition, with at most $k$ cells, of a
sufficiently long arithmetic progression at least one cell contains a
$l$-term arithmetic progression.
For this instance, finite arithmetic progressions is the structure
that is preserved, and the number $W$ is a measure of how large we
need our structure to be to induce this preservation property.
Given this natural number $W$, a bit of thought will show that any
natural number bigger than $W$ is also sufficient to ensure that a
$l$-term arithmetic progression is preserved.
With this observation we can make the following definition.

  \begin{defn}
    For $k$, $l \in \bbN$, the number $W(k,l) \in \bbN$ is the
    \textsl{van der Waerden number} if and only if it is the smallest
    number such that whenever $A \subseteq \bbN$ is a $W(k,l)$-term
    arithmetic progression and $A = \bigcup_{i=1}^k C_i$, there exist
    $i \in \{1, 2, \ldots, k\}$ and $a, d \in \bbN$ such that
      \[
        \{\, a, a+d, \ldots, a+(l-1)d \,\} \subseteq C_i.
      \]
  \end{defn}

Van der Waerden's Theorem and the well-ordering principle of the
natural numbers shows that $W(k,l)$ exists for all $k$, $l \in \bbN$.

  \begin{rmk}
    The attentive reader may have already noticed that
    although we have mentioned finite partitions we have written $A$
    as a finite union.
    Of course, a finite partition is not the same thing as a finite
    union; but, as a practical matter it often doesn't affect the
    truth of Ramsey Theoretic results if we conflate the two. 
    Therefore, we follow the usual convention of writing finite
    partitions and finite unions without bothering to make a
    distinction between the two.
  \end{rmk}

  \begin{defn}
    Let $l \in \bbN$. 
    For $N \in \bbN$ define 
      \begin{align*}
        r_l(N) &= \max\bigr\{\, A \subseteq \{1, 2, \ldots, N\} : 
          \hbox{there does not exist $a$,$b \in \bbN$ such that } \\
          &\hspace{5em}\hbox{$\{\,a, a+d, \ldots, a+(l-1)d \,\}
            \subseteq A$} \bigl\}.
      \end{align*}
  \end{defn}

  \begin{roth}
    $r_3(N) = O\bigl(\frac{N}{\log\log N}\bigl)$.
  \end{roth}
\section{Broader Impact}

\bibliographystyle{plain}
\bibliography{references}
\end{document}

\documentclass[12pt]{article}

\usepackage[margin=1in]{geometry}
\usepackage[doublespacing]{setspace}
\usepackage{amsthm, amssymb, amsmath}

\newtheoremstyle{plain}{3mm}{3mm}{\slshape}{}{\bfseries}{.}{.5em}{}
\theoremstyle{plain}
\newtheorem{thm}{Theorem}[section]
\newtheorem{vdw}[thm]{Van der Waerden's Theorem}

\theoremstyle{definition}
\newtheorem{defn}[thm]{Definition}

\newcommand{\bbN}{\mathbb{N}}
\newcommand{\VDW}{\mathcal{VDW}}


\begin{document}
\section{Research Description}
\subsection{Van der Waerden's Theorem}
The field of Ramsey Theory is broadly described as the study of large
mathematical structures that are, in some sense, preserved under
finite partitions.
From the quantitative viewpoint, Ramsey Theorists investigate
estimates on how ``large'' a particular structure must be to induce
this preservation property.
(Except in a few small cases computing exactly how large a particular
structure must be is beyond current mathematical technology.)
Qualitatively however, the focus is on the properties of structures that are preserved, and how these structures are preserved, under finite partitions.
These two synergistic strands form the guiding principle for most
research in Ramsey Theory.
But, for this research statement we will almost exclusively focus on
the qualitative aspects of the field, and hence for us ``large'' will
almost always simply mean infinite.


Van der Waerden's Theorem is an early\cite{Van-der-Waerden:1927fk}
result in Ramsey Theory that directly illustrates both these
quantitative and qualitative aspects.
Like most important theorems, there are many equivalent ways to state
van der Waerden's Theorem.
Since our focus is on qualitative results, we choose to give a
formulation that emphases the theorem's qualitative character. 
  \begin{vdw}
    For every natural number $r$, if\/ $\bbN = \bigcup_{i=1}^r C_i$,
    then for every natural number $k$ there exist $i \in \{1, 2,
    \ldots, r\}$ and natural numbers $a$ and $d$ such that $\{\, a,
    a+d, \ldots, a+(k-1)d\,\} \subseteq C_i$.
  \end{vdw}
In other words, if we finitely partition the natural numbers, we can
find a finite arithmetic progression of any length in at least one
class of the partition. 
Van der Waerden's Theorem naturally motivates several
qualitative questions. 
To help state our questions we introduce
the following nonstandard but suggestive notation.
  \begin{defn}
    For $A \subseteq \bbN$, let $\VDW(A)$ represent the
    following statement
      \begin{quote}
      For all natural numbers $r$, if $A = \bigcup_{i=1}^r C_i$, then
      for every natural number $k$ there exist $i \in \{1, 2,
      \ldots, r\}$ and natural numbers $a$ and $d$ such that $\{\, a,
      a+d, \ldots, a+(k-1)d\,\} \subseteq C_i$.
     \end{quote}
  \end{defn}
Using this notion we can vaguely ask what are the properties of
subsets $A \subseteq \bbN$ for which $\VDW(A)$ is true?

\bibliographystyle{plain}
\bibliography{references}
\end{document}

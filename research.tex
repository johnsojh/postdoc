\documentclass[12pt]{article}

\usepackage[margin=1in]{geometry}
\usepackage[doublespacing]{setspace}
\usepackage{amsthm, amssymb, amsmath}
\usepackage[all]{xy}

\newtheoremstyle{plain}{3mm}{3mm}{\slshape}{}{\bfseries}{.}{.5em}{}
\theoremstyle{plain}
\newtheorem{thm}{Theorem}[section]
\newtheorem{vdw}[thm]{Van der Waerden's Theorem}
\newtheorem{fst}[thm]{Hindman's Theorem}

\theoremstyle{definition}
\newtheorem{defn}[thm]{Definition}
\newtheorem{rmk}[thm]{Remark}

\newcommand{\bbN}{\mathbb{N}}
\newcommand{\VDW}{\mathcal{VDW}}
\newcommand{\la}{\langle}
\newcommand{\ra}{\rangle}


\begin{document}
\section{Research Description}
\subsection{Hindman's Theorem}
One of the earliest result in Ramsey Theory was proved by Issai Schur
in \cite{Schur:1916fk}.
His theorem states that if the natural numbers are divided into
finitely many classes, then there exists natural numbers $x$ and $y$
such that $x$, $y$, and $x+y$ all lie in one class.
Schur's Theorem has been generalized in a particular direction by
several people.
(Among them are, a student of Schur, Richard Rado in
\cite{Rado:1933kx}, Jon Henry Sanders in \cite{Sanders:1968uq}, and
Jon Folkman is credited in \cite{Graham:1971vn}.)
This particular generalization states that if the natural numbers are
divided into finitely many classes, then for ever natural number $m$
there exists a sequence of natural numbers $\la x_n \ra_{n=1}^m$ such
that the set $\{\, \sum_{n \in F} x_n : \emptyset \ne F \subseteq \{1,
2, \ldots, m\} \,\}$ is contained in one of the classes.
However in \cite{Hindman:1974ys}, Neil Hindman proved the most general
version of these Schur-type theorems currently known. 

  \begin{fst}
    For all natural numbers $r$, if \/ $\bbN = \bigcup_{i=1}^r C_i$, then
    there exist $i \in \{1, 2, \ldots, r\}$ and a sequence of natural
    numbers $\la x_n \ra_{n=1}^\infty$ such that
      \[
        \Bigl\{\, \sum_{n\in F} x_n : \emptyset \ne F \subseteq \bbN
        \hbox{ is finite} \,\Bigr\} \subseteq C_i.
      \]
  \end{fst}

By \cite[Theorems 16.28 and 16.29]{Hindman:1998fk}, Hindman's Theorem
is a strictly stronger then the Rado-Sanders-Folkman result, and
\cite{Milliken:1978fk} shows that there are limits on any possible
uncountable generalization of Hindman's Theorem.
The original proof of this theorem uses an elementary, but
complicated, combinatorial argument. 
Baumgartner in \cite{Baumgartner:1974uq} produced a shorter
combinatorial proof based off the original argument. 
However the easiest, and with the proper background, the shortest
proof involves the somewhat exotic objects called idempotent
ultrafilters. 

The fact that Hindman's Theorem is intimately connected with
idempotent ultrafilters is a fortuitous connection that has proven to
be only the first of the many interesting connections between Ramsey
Theory and ultrafilters.
However before we can explore these applications we review a
construction of the Stone-\v{C}ech compactification of $\bbN$ via
ultrafilters and learn how we can give an algebraic structure to this
compactification.

\subsection{Algebraic Structure of the Stone-\v{C}ech Compactification}
We will construct the Stone-\v{C}ech compactification of $\bbN$ via
ultrafilters of $\bbN$.
  \begin{defn}
    Let $p \subseteq \mathcal{P}(\bbN)$ be nonempty.
    We call $p$ an \textsl{ultrafilter on $\bbN$} if and only if it
    satisfies the following four properties.
      \begin{itemize}
        \item[(1)] If $A \in p$ and $A \subseteq B \subseteq \bbN$,
          then $B \in p$.
        \item[(2)] If $A$, $B \in p$, then $A \cap B \in p$.
        \item[(3)] $\emptyset \not\in p$.
        \item[(4)] For all $A \subseteq \bbN$, either $A \in p$ or
          $\bbN \setminus A \in p$.
      \end{itemize}
  \end{defn}
It's easy to construct (essentially trivial) ultrafilters. 
For all $a \in \bbN$, put $e(a) = \{\, A \subseteq \bbN : a \in A
\,\}$.
Then it is a simple exercise to show that $e(a)$ is an ultrafilter on
$\bbN$. 
We call such ultrafilters \textsl{principal ultrafilters}. 
The existence of nonprincipal ultrafilters follows from an application
of Zorn's Lemma. 

We will soon put a topology and algebraic structure on the collection
of all ultrafilters on $\bbN$. 
In preparation, we make the following list of definitions.

  \begin{defn}
    \begin{itemize}
      \item[(a)] $\beta\bbN = \{\, p : \hbox{$p$ is an ultrafilter on
          $\bbN$} \,\}$.
      \item[(b)] For all $A \subseteq \bbN$, put $\overline{A} = \{\,
        p \in \beta\bbN : A \in p \,\}$.
      \item[(c)] For all $A \subseteq \bbN$ and $x \in \bbN$, put
        $-x+A = \{\, y \in \bbN : x+y \in A \,\}$.
      \item[(d)] For $p$, $q \in \beta\bbN$, define
        $p+q = \bigl\{\, A \subseteq \bbN : \{\, x \in \bbN : -x +A \in q
        \,\} \in p \,\bigr\}$.
      \item[(e)] We say $p \in \beta\bbN$ is an \textsl{idempotent} if
        and only if $p + p = p$.
    \end{itemize}
  \end{defn}

The collection $\{\, \overline{A} : A \subseteq \bbN \,\}$ forms a
basis for a topology on $\beta\bbN$.
If we identify $\bbN$ with the principal ultrafilters by the function
$a \mapsto e(a)$, then $\beta\bbN$ is simply the Stone-\v{C}ech
compactification of $\bbN$. 
For all $p$, $q \in \beta\bbN$, we can check that $p+q$ is an
ultrafilter and hence $p+q \in \beta\bbN$.
In fact with this definition $(\beta\bbN, +)$ becomes a (commutative!)
semigroup, and with our identification of $\bbN$ with the principal
ultrafilters the addition on $\beta\bbN$ is an extension of the usual
addition on $\bbN$. 

With all of our definitions $(\beta\bbN, +)$ becomes a compact
Hausdorff right-topological semigroup. 
(The right-topological semigroup part means that for all $p \in
\beta\bbN$, the function $q \mapsto q+p$ is continuous.)
It's a fact due to Robert Ellis that every compact right-topological
semigroup contains idempotents. 
It's also a particular fact that $\beta\bbN$ has $2^\mathfrak{c}$ (see
\cite[Theorem 6.44]{Hindman:1998fk}) idempotents
where $\mathfrak{c}$ is the cardinality of the continuum. 

\subsection{Van der Waerden's Theorem}
The field of Ramsey Theory is broadly described as the study of large
mathematical structures that are, in some sense, preserved under
finite partitions.
From the quantitative viewpoint, Ramsey Theorists investigate
estimates on how ``large'' a particular structure must be to induce
this preservation property.
(Except in a few small cases computing exactly how large a particular
structure must be is beyond current mathematical technology.)
Qualitatively however, the focus is on the properties of structures that are preserved, and how these structures are preserved, under finite partitions.
These two synergistic strands form the guiding principle for most
research in Ramsey Theory.
But, for this research statement we will almost exclusively focus on
the qualitative aspects of the field, and hence for us ``large'' will
almost always simply mean infinite.


Van der Waerden's Theorem is an early
result in Ramsey Theory that directly illustrates both these
quantitative and qualitative aspects.
Like most important theorems, there are many equivalent ways to state
van der Waerden's Theorem.
Since our focus is on qualitative results, we choose to give a
formulation that emphases the theorem's qualitative character. 
  \begin{vdw}
    
  \end{vdw}
In other words, if we finitely partition the natural numbers, we can
find a finite arithmetic progression of any length in at least one
class of the partition. 
Van der Waerden's Theorem naturally motivates several
qualitative questions. 
To help state our questions we introduce
the following nonstandard but suggestive notation.
  \begin{defn}
    For $A \subseteq \bbN$ and natural numbers $r$ and $k$, let
    $\VDW(A;r,k)$ represent the following statement
      \begin{quote}
        If $A = \bigcup_{i=1}^r C_i$, then there exist $i \in \{1, 2,
        \ldots, r\}$ and natural numbers $a$ and $d$ such that $\{\, a,
        a+d, \ldots, a+(k-1)d\,\} \subseteq C_i$.
     \end{quote}
  \end{defn}
Using this notation van der Waerden's Theorem asserts that for all
natural numbers $r$ and $k$ the statement $\VDW(\bbN;r,k)$ is true.
However, we can be a bit more precise than this!
A moments thought proves that for $A \subseteq \bbN$ and any natural
numbers $r$ and $k$ the following table of implications is valid.
  % Would be nice if I label this table.
  \begin{quote}
    \xymatrix{
      {\VDW(A;r,k)} \ar@{=>}[r]\ar@{=>}[d] & {\VDW(A;r-1,k)}
        \ar@{=>}[r]\ar@{=>}[d] & \cdots \ar@{=>}[r]\ar@{=>}[d] &
        {\VDW(A;1,k)} \ar@{=>}[d] \\
      {\VDW(A;r,k-1)} \ar@{=>}[r]\ar@{=>}[d] & {\VDW(A;r-1,k-1)}
        \ar@{=>}[r]\ar@{=>}[d] & \cdots \ar@{=>}[r]\ar@{=>}[d] &
        {\VDW(A;1,k-1)} \ar@{=>}[d] \\
      \vdots \ar@{=>}[d] & \vdots \ar@{=>}[d] 
        & \vdots \ar@{=>}[d]  & \vdots \ar@{=>}[d] \\
      {\VDW(A;r,1)} \ar@{=>}[r]& {\VDW(A;r-1,1)}\ar@{=>}[r] & 
        \cdots \ar@{=>}[r]& {\VDW(A;1,1)} \\
    }
  \end{quote}
Even though this table was figured out just purely off of the
definition of $\VDW(A; r,k)$, it does give us a hint at how the van
der Waerden's Theorem is proved.
Most proofs of van der Waerden's Theorem comes from the observation
that it suffices to prove for all natural numbers $r$ and $k$ the
following table of implications (reversed from what we had before).
  % Would be nice if I label this table.
  \begin{quote}
    \xymatrix{
      {\VDW(\bbN;r,k)} & {\VDW(\bbN;r-1,k)}
        \ar@{=>}[l] & \cdots \ar@{=>}[l] &
        {\VDW(\bbN;1,k)}\ar@{=>}[l]  \\
      {\VDW(\bbN;r,k-1)} \ar@{=>}[u] & {\VDW(\bbN;r-1,k-1)}
        \ar@{=>}[l]\ar@{=>}[u] & \cdots \ar@{=>}[l]\ar@{=>}[u] &
        {\VDW(\bbN;1,k-1)} \ar@{=>}[u]\ar@{=>}[l] \\
      \vdots \ar@{=>}[u] & \vdots \ar@{=>}[u] 
        & \vdots \ar@{=>}[u]  & \vdots \ar@{=>}[u] \\
      {\VDW(\bbN;r,1)} \ar@{=>}[u]& {\VDW(\bbN;r-1,1)}\ar@{=>}[l]
        \ar@{=>}[u] & 
        \cdots \ar@{=>}[l]\ar@{=>}[u]& {\VDW(\bbN;1,1)}\ar@{=>}[u]
        \ar@{=>}[l] \\
    }
  \end{quote}
Typically most proofs of van der Waerden's Theorem follows from
assuming that one row of the above table is true and proceeds to show
that the first entry in the next row up is ???

Using this notion we can vaguely ask what are the properties of
subsets $A \subseteq \bbN$ for which $\VDW(A)$ is true?

\bibliographystyle{plain}
\bibliography{references}
\end{document}

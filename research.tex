\documentclass[12pt]{article}

\usepackage[margin=1in]{geometry}
\usepackage[doublespacing]{setspace}
\usepackage{amsthm, amssymb, amsmath}

\newtheoremstyle{plain}{3mm}{3mm}{\slshape}{}{\bfseries}{.}{.5em}{}
\theoremstyle{plain}
\newtheorem{thm}{Theorem}[section]


\begin{document}
\section{Research Description}
\subsection{Van der Waerden's Theorem}
The field of Ramsey Theory is broadly described as the study of large mathematical structures that are, in some sense, preserved under finite partitions.
From the quantitative viewpoint, Ramsey Theorists investigate estimates on how ``large'' a particular structure must be to induce this preservation property.
(Except in a few small cases computing exactly how large a particular structure must be is beyond current mathematical technology.)
Qualitatively however, the focus is on the properties of structures that are preserved, and how these structures are preserved, under finite partitions.
These two synergistic strands form the guiding principle for most research in Ramsey Theory.
But, for this research statement we will almost exclusively focus on the qualitative aspects of the field, and hence for us ``large'' will almost always simply mean infinite.

\bibliographystyle{plain}
\bibliography{references}
\end{document}

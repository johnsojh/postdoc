\documentclass[12pt]{article}

\usepackage[margin=1in]{geometry}
\usepackage[doublespacing]{setspace}
\usepackage{amsthm, amssymb, amsmath}
\usepackage[all]{xy}

\newtheoremstyle{plain}{3mm}{3mm}{\slshape}{}{\bfseries}{.}{.5em}{}
\theoremstyle{plain}
\newtheorem{thm}{Theorem}[section]
\newtheorem{vdw}[thm]{Van der Waerden's Theorem}
\newtheorem{fst}[thm]{Hindman's Theorem}

\theoremstyle{definition}
\newtheorem{defn}[thm]{Definition}
\newtheorem{rmk}[thm]{Remark}

\newcommand{\bbN}{\mathbb{N}}
\newcommand{\VDW}{\mathcal{VDW}}
\newcommand{\la}{\langle}
\newcommand{\ra}{\rangle}


\begin{document}
\section{Research Description}
\subsection{Hindman's Theorem}
Issai Schur, in \cite{Schur:1916fk}, proved one of the earliest
results in Ramsey Theory
His theorem states that if the natural numbers are divided into
finitely many classes, then there exists natural numbers $x$ and $y$
such that $x$, $y$, and $x+y$ all lie in one class.
Schur's Theorem has been generalized, in one particular direction, by
several people.
(Among them Richard Rado in \cite{Rado:1933kx}, who was a student of
Schur, Jon Henry Sanders in \cite{Sanders:1968uq}, and Jon Folkman is
credited in \cite{Graham:1971vn}.)
This particular generalization states that if the natural numbers are
divided into finitely many classes, then for every natural number $m$
there exists a sequence of natural numbers $\la x_n \ra_{n=1}^m$ such
that the set $\bigl\{\, \sum_{n \in F} x_n : \emptyset \ne F \subseteq \{1,
2, \ldots, m\} \,\bigr\}$ is contained in one of the classes.
However in \cite{Hindman:1974ys}, Neil Hindman proved the most general
version of these Schur-type theorems currently known. 

  \begin{fst}
    For all natural numbers $r$, if \/ $\bbN = \bigcup_{i=1}^r C_i$, then
    there exist $i \in \{1, 2, \ldots, r\}$ and a sequence of natural
    numbers $\la x_n \ra_{n=1}^\infty$ such that
      \[
        \Bigl\{\, \sum_{n\in F} x_n : \emptyset \ne F \subseteq \bbN
        \hbox{ is finite} \,\Bigr\} \subseteq C_i.
      \]
  \end{fst}

Intuitively Hindman's Theorem is a very strong result.
For instance, it's strictly stronger than the Rado-Sanders-Folkman
result \cite[Theorems 16.28 and 16.29]{Hindman:1998fk}; and,
\cite{Milliken:1978fk} shows that there are limits on any possible
uncountable generalizations of Hindman's Theorem. 
We shall soon see further evidence for the strength of this theorem,
since in a certain sense, Hindman's Theorem touches on most of the
known Ramsey Theory type results. 
This connection was foreshadowed by the search for a proof of this
result. 

The original proof of this theorem uses an elementary, but
complicated, combinatorial argument. 
Baumgartner in \cite{Baumgartner:1974uq} produced a shorter
combinatorial proof, based off the original argument, by looking at an
equivalent statement involving the union of finite subsets of $\bbN$. 
However it was known that an easy proof of Hindman's Theorem would
follow from the existence of the somewhat exotic objects called
idempotent ultrafilters. 
The notes section in \cite[Chapter 5]{Hindman:1998fk} contain a brief
and interesting history about Hindman's Theorem. 

The fact that Hindman's Theorem is intimately connected with
idempotent ultrafilters is a fortuitous connection that has proven to
be only the first of many interesting connections between Ramsey
Theory and ultrafilters.
Moreover, despite over thirty years of study, there are still many
attractive open questions about the applications of ultrafilters to
Ramsey Theory. 
Before we can even state these questions we pause to give a brief
outline on the construction of the algebraic structure of the
Stone-\v{C}ech compactification of $\bbN$ via ultrafilters.

\subsection{Algebraic Structure of the Stone-\v{C}ech Compactification}
We first recall the definition of an ultrafilters on $\bbN$.
  \begin{defn}
    \label{defn:uf}
    Let $p \subseteq \mathcal{P}(\bbN)$ be nonempty.
    We call $p$ an \textsl{ultrafilter on $\bbN$} if and only if it
    satisfies the following four properties.
      \begin{itemize}
        \item[(1)] If $A \in p$ and $A \subseteq B \subseteq \bbN$,
          then $B \in p$.
        \item[(2)] If $A$, $B \in p$, then $A \cap B \in p$.
        \item[(3)] $\emptyset \not\in p$.
        \item[(4)] For all $k \in \bbN$, if $A_i \subseteq \bbN$ for
          all $i \in \{1, 2, \ldots, k\}$, and $\bigcup_{i=1}^k A_i
          \in p$, then there exists $i \in \{1, 2, \ldots, k\}$ such
          that $A_i \in p$.
      \end{itemize}
  \end{defn}
  \begin{rmk}
    We have chosen this, somewhat nonstandard, form of property (4) to
    illustrate the connection between partition regular Ramsey Theory
    statements and ultrafilters.
    \cite[Theorem 5.7]{Hindman:1998fk} makes the connection precise by
    roughly stating that ``any question in Ramsey Theory is a question
    about ultrafilters.''
  \end{rmk}

It's easy to construct (essentially trivial) ultrafilters. 
For all $a \in \bbN$, put $e(a) = \{\, A \subseteq \bbN : a \in A
\,\}$.
Then it is a simple exercise to show that $e(a)$ is an ultrafilter on
$\bbN$. 
We call such ultrafilters \textsl{principal ultrafilters}. 
The existence of \textsl{nonprincipal ultrafilters} follows from an
application of Zorn's Lemma. 

We will soon put a topology and algebraic structure on the collection
of all ultrafilters on $\bbN$. 
In preparation, we make the following list of definitions.

  \begin{defn}
    \label{defn:alg}
    \begin{itemize}
      \item[(a)] $\beta\bbN = \{\, p : \hbox{$p$ is an ultrafilter on
          $\bbN$} \,\}$.
      \item[(b)] For all $A \subseteq \bbN$, put $\overline{A} = \{\,
        p \in \beta\bbN : A \in p \,\}$.
      \item[(c)] For all $A \subseteq \bbN$ and $x \in \bbN$, put
        $-x+A = \{\, y \in \bbN : x+y \in A \,\}$.
      \item[(d)] For $p$, $q \in \beta\bbN$, define
        $p+q = \bigl\{\, A \subseteq \bbN : \{\, x \in \bbN : -x +A \in q
        \,\} \in p \,\bigr\}$.
      \item[(e)] We say $p \in \beta\bbN$ is an \textsl{idempotent
          (ultrafilter)} if  and only if $p + p = p$.
    \end{itemize}
  \end{defn}

The collection $\{\, \overline{A} : A \subseteq \bbN \,\}$ forms a
basis for a topology on $\beta\bbN$.
If we identify $\bbN$ with the principal ultrafilters by the function
$a \mapsto e(a)$, then $\beta\bbN$ is simply the Stone-\v{C}ech
compactification of $\bbN$. 
For all $p$, $q \in \beta\bbN$, we can check that $p+q$ is an
ultrafilter and hence $p+q \in \beta\bbN$.
In fact with this definition $(\beta\bbN, +)$ becomes a (``highly''
noncommutative!) semigroup, and with our identification of $\bbN$ with
the principal ultrafilters the addition on $\beta\bbN$ is an extension
of the usual addition on $\bbN$. 

We can summarize all of our definitions by saying that $(\beta\bbN,
+)$ is a compact Hausdorff right-topological semigroup. 
(The right-topological semigroup part means that for all $p \in
\beta\bbN$, the function $q \mapsto q+p$ is continuous.)
It's a fact due to Robert Ellis that every compact right-topological
semigroup contains idempotents. 
It's also a fact particular to $\beta\bbN$ that this space has
$2^\mathfrak{c}$ (see \cite[Theorem 6.44]{Hindman:1998fk}) idempotents
where $\mathfrak{c}$ is the cardinality of the continuum. 

The reader may well wonder how all of this elaborate equipment can be
put to use. 
As an illustrative example, we indicate how idempotent ultrafilters
can be used to give an easy proof of Hindman's Theorem. 
(We won't prove the whole thing however, we'll just show the $m=3$
case of the Rado-Sanders-Folkman Theorem.)

  \begin{thm}
    For every natural number $r$, if  \/ $\bbN = \bigcup_{i=1}^r C_i$,
    then there exist $i \in \{1, 2, \ldots, r\}$ and a finite sequence
    of natural numbers $\la x_1, x_2, x_3 \ra$ such that $\bigl\{\,
    \sum_{n\in F} x_n : \emptyset \ne F \subseteq \{1,2,3\} \,\bigr\}
    \subseteq C_i$.
  \end{thm}
  \begin{proof}
    Pick an idempotent ultrafilter $p + p = p \in \beta\bbN$. 
    Since $\bbN = \bigcup_{i=1}^r C_i$ and $\bbN \in p$, it follows by
    Definition \ref{defn:uf} that we can pick $i \in \{1, 2, \ldots,
    r\}$ such that $C_i \in p$. 
    By Definition \ref{defn:alg}(d) we have that $\{\, x \in \bbN : -x
    + C_i \in p \,\} \in p$.
    Hence again by definition \ref{defn:uf}, we have that $\{\, x \in
    C_i : -x + C_i \in p \,\} = C_i \cap
    \{\, x \in \bbN : -x + C_i \in p \,\} \in p$.

    Pick $x_1 \in \{\,x \in C_i: -x + C_i \in p \,\}$, then
    $x_1 \in C_i$ and $-x_1 + C_i \in p$. 
    We have that $\{\,x \in \bbN : -(x_1+x)+C_i \in p \,\} \in p$.
    So pick $x_2 \in \{\, x \in \bbN : -(x_1+x)+C_i \in p \,\} \cap
    (-x_1+C) \cap \{\, x \in C_i : -x+C_i \in p \,\}$.
    Then $x_2 \in C_i$, $x_1+x_2 \in C_i$, $-x_2 + C_i \in p$, and
    $-(x_1 + x_2) + C_i \in p$.
    Finally pick $x_3 \in C_i \cap (-x_1+C_i) \cap (-x_2+C_i) \cap
    -(x_1+x_2) + C_i$. 
    Hence we have that $\{\, x_1, x_2, x_3, x_1+x_2, x_1+x_3, x_2+x_3,
    x_1+x_2+x_3 \,\} \subseteq C_i$.
  \end{proof}
Turning this into a full proof of Hindman's Theorem requires a
suitable induction argument on the length of the sequence you have
constructed so far. 

To finish up with our algebraic preliminaries we need to make one more
set of definitions.
With this next set of definitions we will be able to state most
the currently known Ramsey Theory results in terms of
the algebraic structure of $\beta\bbN$.
  \begin{defn}
    \begin{itemize}
      \item[(a)] Given subsets $A$ and $B$ of $\beta\bbN$, we put $A +
        B = \{\, p+q : \hbox{$p \in A$ and $q \in B$} \,\}$.
      \item[(b)] A nonempty subset $L \subseteq \beta\bbN$ is called a
        \textsl{left ideal (of $\beta\bbN$)} if and only if $\beta\bbN
        + L \subseteq L$.
      \item[(c)] A nonempty subset $R \subseteq \beta\bbN$ is called a
        \textsl{right ideal (of $\beta\bbN$)} if and only if $R +
        \beta\bbN \subseteq R$.
      \item[(d)] A nonempty subset $I \subseteq \beta\bbN$ is called a
        \textsl{(two-sided) ideal (of $\beta\bbN$)} if and only if $I$
        is both a left and right ideal.
      \item[(e)] We denote the \textsl{smallest (with respect to
          inclusion) ideal of $\beta\bbN$} by $K(\beta\bbN)$.
    \end{itemize}
  \end{defn}
It is a fact\cite[Theorem 2.8]{Hindman:1998fk} that the smallest ideal
exists for any compact Haudorff right-topological semigroup. 


\bibliographystyle{plain}
\bibliography{references}
\end{document}

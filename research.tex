% Draft of my Research Proposal for the Simons Postdoctoral Fellowship
% at Harvard University
% by: John H. Johnson
% email: john.j.jr@gmail.com

\documentclass[12pt]{article}

\usepackage{amsthm, amssymb, amsmath}
% \usepackage[doublespacing]{setspace}

\newtheoremstyle{plain}{3mm}{3mm}{\slshape}{}{\bfseries}{.}{.5em}{}
\theoremstyle{plain}
\newtheorem*{vdw}{Van Der Waerden's Theorem}
\newtheorem*{roth}{Roth's Theorem}
\newtheorem*{sz}{Szemer\'{e}di's Theorem}
\newtheorem*{cst}{Central Sets Theorem}
\newtheorem*{thm}{Theorem}
\newtheorem*{cor}{Corollary}
\newtheorem*{prop}{Proposition}
\newtheorem*{lem}{Lemma}
\newtheorem*{ques}{Question}
\newtheorem*{conj}{Conjecture}
\newtheorem*{fact}{Fact}

\theoremstyle{definition}
\newtheorem*{defn}{Definition}
\newtheorem*{example}{Example}
\newtheorem*{rmk}{Remark}

\newcommand{\la}{\langle}
\newcommand{\ra}{\rangle}
\newcommand{\bbN}{\mathbb{N}}
\newcommand{\setfunc}[2]{\hbox{${}^{\hbox{$#1$}}\hskip -3 pt #2$}}

\begin{document}
\section{Research Summary}

\section{Research Description}
\subsection{Van der Waerden's Theorem}
The field of Ramsey Theory is broadly described as the study of large
mathematical structures that are preserved under finite partitions.
From the qualitative viewpoint, Ramsey Theorists investigate precisely
which structures are preserved, and how they are preserved under
finite partitions.
Quantitatively however, the focus is on computing estimates on how
``large'' a certain structure must be to achieve this preservation
property. 
(Except in a few small cases, computing exactly how large a structure
must be is beyond current mathematical technology.)
These two synergistic strands form the guiding principle for most
research in Ramsey Theory.
For this research statement, we will exclusively focus on the
qualitative aspects of the field, and for us ``large'' will
be simply infinite.

The following is a typical qualitative question.
  \begin{ques}
    For which subsets $A \subseteq \bbN$ is the following property
    true?
      \begin{quote}
        For all natural numbers $r$, if $A = \bigcup_{i=1}^r C_i$,
        then for every natural numbers $k$ there exist $i \in \{1, 2,
        \ldots, r\}$ and natural numbers $a$ and $d$ such that 
          \[
            \{\, a, a+d, \ldots, a+(k-1)d \,\} \subseteq C_i.
          \]
      \end{quote}
  \end{ques}
Van der Waerden's Theorem is an early result
\cite{Van-der-Waerden:1927fk} in Ramsey Theory that partially answers
this question.
This theorem states that if we partition the natural numbers into finitely many
classes then we can find a finite arithmetic progression of any length
in at least one class. 
More precisely we have the following. 
  \begin{vdw}
    For all natural numbers $r$, if $\bbN = \bigcup_{i=1}^r C_i$,
    then for every natural numbers $k$ there exist $i \in \{1, 2,
    \ldots, r\}$ and natural numbers $a$ and $d$ such that 
      \[
        \{\, a , a+d, \ldots, a+(k-1)d \,\} \subseteq C_i.
      \]
  \end{vdw}
It's possible to obtain a more detailed understanding of our question
by studying $(\bbN, +)$ from an algebraic viewpoint. 
At first glance, $(\bbN, +)$ has only a paucity of algebraic
structure. 
However, we shall soon see that we can embed $(\bbN, +)$ into a rich
algebraic structure that can shed light on qualitative questions in
Ramsey Theory. 

\subsection{Algebraic Preliminaries}
The algebraic structure that we shall setup is through semigroup
theory.
We use the following definitions.
  \begin{defn}
    Let $(S, \cdot)$ be a pair such that $S$ is nonempty and $\cdot$ is a
    binary operation on $S$.
      \begin{itemize}
        \item[(a)] $S$ is a \textsl{semigroup} if and only if for all
          $a$, $b$, and $c \in S$, we have $a \cdot (b \cdot c) = (a
          \cdot b) \cdot c$.
        \item[(b)] For all $a \in S$, define the functions $\lambda_a
          : S \to S$ and $\rho_a : S \to S$ by $\lambda_a(x) = a\cdot
          x$ and $\rho_a(x) = x\cdot a$, respectively.
        \item[(c)] Let $\emptyset \ne I \subseteq S$.
          We say $I$ is a \textsl{left ideal of $S$} if and only if
          $S\cdot I \subseteq I$.
          We say $I$ is a \textsl{right ideal of $S$} if and only if
          $S\cdot I \subseteq I$.
          We say $I$ is a \textsl{(two-sided) ideal of $S$} if and only if
          $I$ is both a left ideal and right ideal of $S$.
      \end{itemize}
  \end{defn}
Even looking at $(\bbN, +)$ as a semigroup shows that this set has a
pretty bare algebraic structure. 
(For instance infinite segments of $\bbN$ are the only ideals of this
semigroup.)
To make $(\bbN, +)$ algebraically more useful we will look at the
Stone-\v{C}ech compactification $\beta\bbN$ of $\bbN$ and extend the
usual addition of $\bbN$ to $\beta\bbN$.

  \begin{defn}
    Let $\mathcal{U} \subseteq \mathcal{P}(\bbN)$.
    We say $\mathcal{U}$ is an \textsl{ultrafilter on $\bbN$} if and
    only if it has the following four properties:
      \begin{itemize}
        \item[(1)] $\mathcal{U}$ is nonempty.
        \item[(2)] $\emptyset \not\in \mathcal{U}$. 
        \item[(3)] If $A \in \mathcal{U}$ and $A \subseteq B \subseteq
          \bbN$, then $B \in \mathcal{U}$.
        \item[(4)] For all $A \subseteq \bbN$, then either $A \in
          \mathcal{U}$ or $\bbN \setminus A \in \mathcal{U}$.
      \end{itemize}
  \end{defn}

\subsection{Viewing van der Waerden's Theorem Algebraically}
\subsection{Van der Waerden's Theorem and Algebraic Structure}
Like most important theorems, there are many equivalent ways to state
van der Waerden's Theorem.
Since we will be interested in the qualitative aspects of Ramsey
Theory, we choose to give a formulation that emphasizes this viewpoint.


In order words, 
Qualitatively we are interested in the following two questions.
  \begin{ques}
    \vspace*{1em}
    \begin{itemize}
      \item[(1)] For which subsets $A \subseteq \bbN$ is it true that
        for all positive integers $r$, if $A = \bigcup_{i=1}^r C_i$,
        then for every positive integer $k$ there exist $i \in \{1, 2,
        \ldots, r \}$ and positive integers $a$ and $b$ such that 
          \[
            \{\, a, a+d, \ldots, a+(k-1)d \,\} \subseteq C_i?
          \]
       \item[(2)] Is there a ``minimal'' subset $A \subseteq \bbN$
         such that whenever we partition $A$ into finitely many
         classes we can find a finite arithmetic progression of any
         length in at least one class?
    \end{itemize}
  \end{ques}

It is possible to study both of these questions from the algebraic
viewpoint. 
We provide a brief review of this 


A bit of thought proves that any
natural number bigger than $W$ is sufficient to ensure that a
$l$-term arithmetic progression is preserved.
Using this small observation and the well-ordering principle for
natural numbers enables us to make the following (nonvacuous) definition.

  \begin{defn}
    For $k$, $l \in \bbN$, the number $W(k,l) \in \bbN$ is the
    \textsl{van der Waerden number} if and only if it is the smallest
    number such that whenever $A \subseteq \bbN$ contains a $W(k,l)$-term
    arithmetic progression and $A = \bigcup_{i=1}^k C_i$, there exist
    $i \in \{1, 2, \ldots, k\}$ and $a, d \in \bbN$ such that
      \[
        \{\, a, a+d, \ldots, a+(l-1)d \,\} \subseteq C_i.
      \]
  \end{defn}

  \begin{rmk}
    The attentive reader may have already noticed that
    although we have mentioned finite partitions we have written $A$
    as a finite union.
    Of course, a finite partition is not the same thing as a finite
    union; but, as a practical matter it often doesn't affect the
    truth of Ramsey Theoretic results if we conflate the two. 
    Therefore we follow the usual convention of writing finite
    partitions and finite unions without bothering to make a
    distinction between the two.
  \end{rmk}

%   \begin{defn}
%     Let $l \in \bbN$. 
%     For $N \in \bbN$ define 
%       \begin{align*}
%         r_l(N) &= \max\bigr\{\, A \subseteq \{1, 2, \ldots, N\} : 
%           \hbox{there does not exist $a$,$b \in \bbN$ such that } \\
%           &\hspace{5em}\hbox{$\{\,a, a+d, \ldots, a+(l-1)d \,\}
%             \subseteq A$} \bigl\}.
%       \end{align*}
%   \end{defn}

%   \begin{roth}
%     $r_3(N) = O\bigl(\frac{N}{\log\log N}\bigl)$.
%   \end{roth}
% \section{Broader Impact}

\bibliographystyle{plain}
\bibliography{references}
\end{document}

\documentclass[12pt]{article}

\usepackage[margin=1in]{geometry}
\usepackage[doublespacing]{setspace}
\usepackage{amsthm, amssymb, amsmath}
\usepackage[all]{xy}

\newtheoremstyle{plain}{3mm}{3mm}{\slshape}{}{\bfseries}{.}{.5em}{}
\theoremstyle{plain}
\newtheorem{thm}{Theorem}[section]
\newtheorem{vdw}[thm]{Van der Waerden's Theorem}
\newtheorem{fst}[thm]{Hindman's Theorem}
\newtheorem{schur}[thm]{Schur's Theorem}

\theoremstyle{definition}
\newtheorem{defn}[thm]{Definition}
\newtheorem{rmk}[thm]{Remark}

\newcommand{\bbN}{\mathbb{N}}
\newcommand{\VDW}{\mathcal{VDW}}
\newcommand{\la}{\langle}
\newcommand{\ra}{\rangle}


\begin{document}
\section{Research Description}
\subsection{Hindman's Theorem}
Issai Schur, in \cite{Schur:1916fk}, proved one of the earliest
results in Ramsey Theory. 
His theorem states that if the natural numbers are divided into
finitely many classes, then there exists natural numbers $x$ and $y$
such that $x$, $y$, and $x+y$ all lie in one class.
Schur's Theorem has been generalized, in one particular direction, by
several people.
(Among them Richard Rado in \cite{Rado:1933kx}, who was a student of
Schur, Jon Henry Sanders in \cite{Sanders:1968uq}, and Jon Folkman is
credited in \cite{Graham:1971vn}.)
This particular generalization states that if the natural numbers are
divided into finitely many classes, then for every natural number $m$
there exists a sequence of natural numbers $\la x_n \ra_{n=1}^m$ such
that the set $\bigl\{\, \sum_{n \in F} x_n : \emptyset \ne F \subseteq \{1,
2, \ldots, m\} \,\bigr\}$ is contained in one of the classes.
However in \cite{Hindman:1974ys}, Neil Hindman proved the most general
version of these Schur-type theorems currently known. 

  \begin{fst}
    For all natural numbers $r$, if \/ $\bbN = \bigcup_{i=1}^r C_i$, then
    there exist $i \in \{1, 2, \ldots, r\}$ and a sequence of natural
    numbers $\la x_n \ra_{n=1}^\infty$ such that
      \[
        \Bigl\{\, \sum_{n\in F} x_n : \emptyset \ne F \subseteq \bbN
        \hbox{ is finite} \,\Bigr\} \subseteq C_i.
      \]
  \end{fst}

Intuitively Hindman's Theorem is a very strong result.
For instance, it's strictly stronger than the Rado-Sanders-Folkman
result \cite[Theorems 16.28 and 16.29]{Hindman:1998fk}; and,
\cite{Milliken:1978fk} shows that there are limits on any possible
uncountable generalizations of Hindman's Theorem. 
We shall soon see further evidence for the strength of this theorem,
since in a certain sense, Hindman's Theorem touches on most of the
known Ramsey Theory type results. 
This connection was foreshadowed by the search for a proof of this
result. 

The original proof of this theorem uses an elementary, but
complicated, combinatorial argument. 
By looking at an equivalent statement, involving the union of finite
subsets of $\bbN$, Baumgartner in \cite{Baumgartner:1974uq} produced a
shorter and slightly easier combinatorial proof based off the original
argument.
However, even before these combinatorial proofs were discovered
it was shown by Fred Galvin that an easy proof of Hindman's Theorem
would follow from the existence of the somewhat exotic objects called
idempotent ultrafilters. 
The notes section in \cite[Chapter 5]{Hindman:1998fk} contain a brief
and interesting history about Hindman's Theorem. 

The fact that Hindman's Theorem is intimately connected with
idempotent ultrafilters is a fortuitous connection that has proven to
be only the first of many interesting connections between Ramsey
Theory and ultrafilters.
(It's a fact that Hindman's Theorem is equivalent to the existence of
idempotent ultrafilters!)
Moreover, despite over thirty years of study, there are still many
attractive open questions about the applications of ultrafilters to
Ramsey Theory. 
Before we can even state these questions we to give a brief
outline on the construction of the algebraic structure of the
Stone-\v{C}ech compactification of $\bbN$ via ultrafilters.

\subsection{Algebraic Structure of the Stone-\v{C}ech Compactification}
We first recall the definition of an ultrafilter on $\bbN$.
(The reader can find a more expository introduction to ultrafilters
and their applications in \cite{Komjath:2008fk}, \cite{Gowers:zr}, or
\cite{Tao:2007uq}.)
  \begin{defn}
    \label{defn:uf}
    Let $p \subseteq \mathcal{P}(\bbN)$ be nonempty.
    We call $p$ an \textsl{ultrafilter on $\bbN$} if and only if it
    satisfies the following four properties:
      \begin{itemize}
        \item[(1)] $\emptyset \not\in p$.
        \item[(2)] If $A$, $B \in p$, then $A \cap B \in p$.
        \item[(3)] If $A \in p$ and $A \subseteq B \subseteq \bbN$,
          then $B \in p$.
        \item[(4)] Let $k \in \bbN$ and for each $i \in \{1, 2,
          \ldots, k\}$ suppose we have $A_i \subseteq \bbN$. 
          If $\bigcup_{i=1}^k A_i \in p$, then there exists $i \in
          \{1, 2, \ldots, k\}$ such that $A_i \in p$.
      \end{itemize}
  \end{defn}
  \begin{rmk}
    For property (4), we have chosen this, somewhat nonstandard, form
    to hint at the connection between Ramsey Theory and ultrafilters.
    In \cite[Theorem 5.7]{Hindman:1998fk} this connection is made
    precise by roughly stating that ``any question in Ramsey Theory is
    a question about ultrafilters.''
  \end{rmk}

It's easy to construct, essentially trivial, ultrafilters. 
For all $a \in \bbN$, put $e(a) = \{\, A \subseteq \bbN : a \in A
\,\}$.
A simple check shows that $e(a)$ is an ultrafilter on
$\bbN$. 
We call such ultrafilters \textsl{principal ultrafilters}. 
The existence of \textsl{nonprincipal ultrafilters} follows from an
application of Zorn's Lemma. 
(Indeed the existence of nonprincipal ultrafilters requires some form
of choice.)

Next, we will put a topology and algebraic structure on the collection
of all ultrafilters on $\bbN$. 
Accordingly, we make the following list of definitions.

  \begin{defn}
    \label{defn:alg}
    \begin{itemize}
      \item[(a)] $\beta\bbN = \{\, p \subseteq \mathcal{P}(\bbN) :
        \hbox{$p$ is an ultrafilter on
        $\bbN$} \,\}$.
      \item[(b)] For all $A \subseteq \bbN$, put $\overline{A} = \{\,
        p \in \beta\bbN : A \in p \,\}$.
      \item[(c)] For all $A \subseteq \bbN$ and $x \in \bbN$, put
        $-x+A = \{\, y \in \bbN : x+y \in A \,\}$.
      \item[(d)] For $p$, $q \in \beta\bbN$, put
        $p+q = \bigl\{\, A \subseteq \bbN : \{\, x \in \bbN : -x +A \in q
        \,\} \in p \,\bigr\}$.
      \item[(e)] We say $p \in \beta\bbN$ is an \textsl{idempotent
          (ultrafilter)} if  and only if $p + p = p$.
    \end{itemize}
  \end{defn}

The collection $\{\, \overline{A} : A \subseteq \bbN \,\}$ forms a
basis for a topology on $\beta\bbN$.
We identify $\bbN$ with the principal ultrafilters by the function
$a \mapsto e(a)$. Then $\beta\bbN$ is simply the Stone-\v{C}ech
compactification of $\bbN$. 
For all $p$, $q \in \beta\bbN$, the sum $p+q \in \beta\bbN$, moreover,
$(\beta\bbN, +)$ is a semigroup where the addition on $\beta\bbN$ is
an extension of the usual addition on $\bbN$. 
(However it is somewhat disconcerting to learn that $(\beta\bbN, +)$
is a ``highly'' noncommutative semigroup, since it contains a copy of
the free group on $2^{2^{\aleph_0}}$ generators \cite[Corollary
7.36]{Hindman:1998fk}.)

In fact with this definition $(\beta\bbN, +)$ becomes a (``highly''
noncommutative!) semigroup, and with our identification of $\bbN$ with
the principal ultrafilters the addition on $\beta\bbN$ is an extension
of the usual addition on $\bbN$. 

We summarize all of our definitions by saying that $(\beta\bbN,
+)$ is a compact Hausdorff right-topological semigroup. 
The right-topological semigroup part means that for all $p \in
\beta\bbN$, the function $q \mapsto q+p$ is continuous.
It's a fact due to Robert Ellis \cite[Corollary 2.10]{Ellis:1969zr}
that every compact right-topological semigroup contains idempotents. 
In the case of $\beta\bbN$ we have 
$2^{2^\aleph_0}$) idempotents \cite[Theorem
6.44]{Hindman:1998fk}. 
Also observe the trivial fact that all idempotents in $\beta\bbN$ are
necessarily noncprincipal ultrafilters. 

To demonstrate how all of this elaborate equipment can be put to use,
we give an easy proof of Schur's Theorem using an idempotent ultrafilter.

  \begin{schur}
    For every natural number $r$, if  \/ $\bbN = \bigcup_{i=1}^r C_i$,
    then there exist $i \in \{1, 2, \ldots, r\}$ and natural numbers
    $x$ and $y$ such that $\{\, x, y, x+y \,\} \subseteq C_i$.
  \end{schur}
  \begin{proof}
    Pick an idempotent ultrafilter $p + p = p \in \beta\bbN$. 
    Since $\bbN = \bigcup_{i=1}^r C_i$ and $\bbN \in p$, it follows by
    Definition \ref{defn:uf}(4) that we can pick $i \in \{1, 2, \ldots,
    r\}$ such that $C_i \in p$. 
    By Definition \ref{defn:alg}(d) we have that $\{\, z \in \bbN : -z
    + C_i \in p \,\} \in p$.
    Hence again by Definition \ref{defn:uf}(2), we have that $\{\, z \in
    C_i : -z + C_i \in p \,\} = C_i \cap
    \{\, z \in \bbN : -z + C_i \in p \,\} \in p$.

    Pick $x \in \{\,z \in C_i: -z + C_i \in p \,\}$, then by
    Definition \ref{defn:uf}(2), $C_i \cap (-x+C_i) \in p$.
    Pick $y \in C_i \cap (-x+C_i)$.
    Then $\{\, x, y, x+y\,\} \subseteq C_i$.
  \end{proof}
For the readers that know that members of nonprincipal ultrafilters
are infinite, a little thought reveals that our proof actually can
produce an infinite collection of pairs $\{\,x,y\,\}$ such that
$\{\,x, y, x+y \,\} \subseteq C_i$. 
Slightly more substantial modifications of our proof, mostly involving
notation and induction, are needed to produce a full proof of
Hindman's Theorem. 
The details of such a proof can be found in \cite[Theorem
5.8]{Hindman:1998fk}.

To finish up with our algebraic preliminaries we make one more
set of definitions.
  \begin{defn}
    \begin{itemize}
      \item[(a)] Given subsets $A$ and $B$ of $\beta\bbN$, we put $A +
        B = \{\, p+q : \hbox{$p \in A$ and $q \in B$} \,\}$.
      \item[(b)] A nonempty subset $L \subseteq \beta\bbN$ is called a
        \textsl{left ideal (of $\beta\bbN$)} if and only if $\beta\bbN
        + L \subseteq L$.
      \item[(c)] A nonempty subset $R \subseteq \beta\bbN$ is called a
        \textsl{right ideal (of $\beta\bbN$)} if and only if $R +
        \beta\bbN \subseteq R$.
      \item[(d)] A nonempty subset $I \subseteq \beta\bbN$ is called a
        \textsl{(two-sided) ideal (of $\beta\bbN$)} if and only if $I$
        is both a left and right ideal.
      \item[(e)] We denote the \textsl{smallest closed(with respect to
          inclusion) ideal of $\beta\bbN$} by $\Delta_1$.
    \end{itemize}
  \end{defn}
It follows from \cite[Theorem 2.8]{Hindman:1998fk} that the smallest ideal
exists for any compact Haudorff right-topological semigroup. 

Using the concepts of closed left ideals and closed two-sided ideals
we can give the connection between another class of Ramsey Theory
statements, and give the open problems I'm interested in researching.

\subsection{Algebraic Proof of Roth's Theorem}
Bartel Leendert van der Waerden proved in
\cite{Van-der-Waerden:1927fk} that if the natural numbers are divided
into finitely many classes, then for every natural number $k$ at least
once class contains a $k$-term arithmetic progression. 
Using van der Waerden's Theorem and the pigeonhole principle it's
possible to prove, the superficially stronger statement, that one
class contains a $k$-term arithmetic progression for every natural
number $k$. 
Almost a decade after van der Waerden published his proof, Erd\H{o}s
and Turan started to investigate which notion of ``size'' guarantees
the conclusion of van der Waerden's Theorem in \cite{Erdos:1936fk}.
To state one of there implicit conjectures in the paper we introduce
the following definition.

  \begin{defn}
    For all $A \subseteq \bbN$, define the \textsl{upper asymptotic
      density} of $A$ by 
      \[
        \overline{d}(A) = \limsup_{N\to\infty} \frac{A \cap \{1, 2,
          \ldots, N\}}{N}.
      \]
  \end{defn}

Erd\H{o}s and Turan conjectured that if $\overline{d}(A) > 0$, then
$A$ contains a $k$-term arithmetic progression for every natural
number $k$.
Their conjecture turned out to be correct and was proved almost forty
years later by Szemer\'{e}di in \cite{Szemeredi:1975uq}. 
To state Szemer\'{e}di's Theorem in algebraic language we introduce
the following two definitions.
  \begin{defn}
    \begin{itemize}
      \item[(a)] $\Delta = \{\, p \in \beta\bbN : \overline{d}(A) > 0
        \hbox{ for all } A \in p \,\}$.
      \item[(b)] $\mathcal{AP} = \{\, p \in \in \beta\bbN : \hbox{for
          all $A \in p$ } A \hbox{ contains a $k$-term arithmetic
          progression for every $k \in \bbN$} \,\}$.
    \end{itemize}
  \end{defn}

Two important results in Ramsey Theory are Szem\'{e}rdi's Theorem and
Roth's Theorem. 
To state these theorems we introduce the familiar notion of \textsl{upper
asymptotic density}
  \begin{defn}
    Given $A \subseteq \bbN$, define $\overline{d}(A) = \limsup_{N \to
      \infty} (A \cap \{1, 2, \ldots, N\})/ N$.
  \end{defn}
Szem\'{e}rdi's Theorem states that if $\overline{d}(A) > 0$, then $A$
contains arbitrarily long arithmetic progressions. 
This particular theorem is hard to prove and has a long history. 
Van der Waerden proved $\la\hbox{blah}\ra$, then Erd\H{o}s and Turan
was interested in which ``size'' of guarantees the conclusion to van
der Waerden's Theorem, they correctly conjectured that if
$\overline{d}(A) >0$, then $A$ contains arbitrarily long arithmetic
progressions. 
The proof of this conjecture was not immediate put proceeded in
steps. 
The first big step was provided by Roth in \cite{Roth:1953fk}. 
In this paper he proved the following $\la\hbox{Roth's Theorem}\ra$.
Next Szem\'{e}rdi gave a combinatorial proof of $\la\hbox{Roth's
  Theorem}\ra$ and extended it to a 4-term arithmetic progression. 
Finally Szem\'{e}rdi was able to prove the full case in
\textsc{SzPaper}. 
Soon after another proof of Szem\'{e}rd's Theorem was provided by
Fursternburg in \textsc{ERTpaper} that used Ergodic Theory. 
Since then several more proofs of Szem\'{e}di's Theorem have appeared
\textsc{cite all of the various version included DHJ}. 
It's an empirical fact that most Ramsey Theory statements are either
about idempotents or ideals.
For instance, van \dots


\bibliographystyle{plain}
\bibliography{references}
\end{document}

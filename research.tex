\documentclass[12pt]{article}

\usepackage[margin=1in]{geometry}
\usepackage[doublespacing]{setspace}
\usepackage{amsthm, amssymb, amsmath}
% \usepackage[all]{xy}
\usepackage{booktabs}

\newtheoremstyle{plain}{3mm}{3mm}{\slshape}{}{\bfseries}{.}{.5em}{}
\theoremstyle{plain}
\newtheorem{thm}{Theorem}[section]
\newtheorem{vdw}[thm]{Van der Waerden's Theorem}
\newtheorem{fst}[thm]{Hindman's Theorem}
\newtheorem{schur}[thm]{Schur's Theorem}
\newtheorem{sz}[thm]{Szemer\'{e}di's Theorem}

\theoremstyle{definition}
\newtheorem{defn}[thm]{Definition}
\newtheorem{rmk}[thm]{Remark}

\newcommand{\bbN}{\mathbb{N}}
\newcommand{\VDW}{\mathcal{VDW}}
\newcommand{\la}{\langle}
\newcommand{\ra}{\rangle}


\begin{document}
\section{Research Description}
\subsection{Hindman's Theorem}
Issai Schur, in \cite{Schur:1916fk}, proved one of the earliest
results in Ramsey Theory. 
His theorem states that if the natural numbers are divided into
finitely many classes, then there exists natural numbers $x$ and $y$
such that $\{\, x, y, x+y \,\}$ is contained in one class.
Richard Rado (a student of Schur) in \cite{Rado:1933kx}, Jon Sanders
in \cite{Sanders:1968uq}, and Jon Folkman (credited in
\cite{Graham:1971vn}) all proved the following generalization of
Schur's Theorem.
If the natural numbers are divided into finitely many classes, then
for every $m \in \bbN$ there exist a sequence $\la x_n \ra_{n=1}^m$
and a class that contains $\bigl\{\, \sum_{n \in F} x_n : \emptyset
\ne F\subseteq \{1, 2, \ldots, m\} \,\bigr\}$.
Several years later Neil Hindman proved, in \cite{Hindman:1974ys},
another generalization of Schur's Theorem which is strictly stronger
than the Rado-Sanders-Folkman Theorem \cite[Theorems 16.28 and
16.29]{Hindman:1998fk} and moreover, cannot be generalized to an
``uncountable index'' \cite{Milliken:1978fk}.
  \begin{fst}
    For all natural numbers $r$, if \/ $\bbN = \bigcup_{i=1}^r C_i$, then
    there exist $i \in \{1, 2, \ldots, r\}$ and a sequence $\la x_n
    \ra_{n=1}^\infty$ in $\bbN$ such that
      \[
        \Bigl\{\, \sum_{n\in F} x_n : \emptyset \ne F \subseteq \bbN
        \hbox{ is finite} \,\Bigr\} \subseteq C_i.
      \]
  \end{fst}

The original proof of this theorem uses an elementary, but
complicated, combinatorial argument. 
(However, Baumgartner gives a shorter and slightly easier
combinatorial proof in \cite{Baumgartner:1974uq}.)
Even before the first combinatorial proof was discovered it was shown
by Fred Galvin that an easy proof of Hindman's Theorem would follow
from the existence of, the somewhat exotic object called, an idempotent
ultrafilter. 
A brief summary of the interesting history of Hindman's Theorem can be
found in the notes section of \cite[Chapter 5]{Hindman:1998fk}. 

This connection between Hindman's Theorem and idempotent ultrafilters
was only the first of many connections between Ramsey Theory and
ultrafilters.
Moreover, despite over thirty years of study, there are still many
attractive open questions on the applications of ultrafilters to
Ramsey Theory. 
Before we can even state those particular questions that interest us,
we pause to give a brief outline of the requisite results and
definitions. 

\subsection{Idempotents in the Stone-\v{C}ech Compactification}
  \begin{defn}
    \label{defn:uf}
    Let $\emptyset \ne p \subseteq \mathcal{P}(\bbN)$.
    We call $p$ an \textsl{ultrafilter on $\bbN$} if and only if it
    satisfies the following four properties:
      \begin{itemize}
        \item[(1)] $\emptyset \not\in p$.
        \item[(2)] If $A$, $B \in p$, then $A \cap B \in p$.
        \item[(3)] If $A \in p$ and $A \subseteq B \subseteq \bbN$,
          then $B \in p$.
        \item[(4)] Let $r \in \bbN$ and for each $i \in \{1, 2,
          \ldots, r\}$ suppose we have $A_i \subseteq \bbN$. 
          If $\bigcup_{i=1}^r A_i \in p$, then there exists $i \in
          \{1, 2, \ldots, r\}$ such that $A_i \in p$.
      \end{itemize}
  \end{defn}
  \begin{rmk}
    For property (4), we have chosen this, somewhat nonstandard, form
    to hint at the connection between Ramsey Theory and ultrafilters.
    In \cite[Theorem 5.7]{Hindman:1998fk} this connection is made
    precise by roughly stating that ``any question in Ramsey Theory is
    a question about ultrafilters.''
    The reader can find a good and quick introduction to ultrafilters
    in the article \cite{Komjath:2008fk}.
  \end{rmk}

It's easy to construct, essentially trivial, ultrafilters. 
For all $a \in \bbN$, put $e(a) = \{\, A \subseteq \bbN : a \in A
\,\}$, then a simple check shows that $e(a)$ is an ultrafilter on $\bbN$.
We call such ultrafilters \textsl{principal ultrafilters}. 
Ultrafilters that are not principal are called \textsl{nonprincipal
  ultrafilters}, and the existence of such ultrafilters follows from
Zorn's Lemma. 

Our next set of definitions will enable us to put a topology and
algebraic structure on the collection of all ultrafilters on $\bbN$.
This will give the Stone-\v{C}ech compactification $\beta\bbN$ the
structure of a compact Hausdorff right-topological semigroup. 
(We will explain what ``right-topological'' means shortly.)

  \begin{defn}
    \label{defn:alg}
    \begin{itemize}
      \item[(a)] $\beta\bbN = \{\, p \subseteq \mathcal{P}(\bbN) :
        \hbox{$p$ is an ultrafilter on
        $\bbN$} \,\}$.
      \item[(b)] For all $A \subseteq \bbN$, put $\overline{A} = \{\,
        p \in \beta\bbN : A \in p \,\}$.
      \item[(c)] For all $A \subseteq \bbN$ and $x \in \bbN$, put
        $-x+A = \{\, y \in \bbN : x+y \in A \,\}$.
      \item[(d)] For $p$, $q \in \beta\bbN$, put
        $p+q = \bigl\{\, A \subseteq \bbN : \{\, x \in \bbN : -x +A \in q
        \,\} \in p \,\bigr\}$.
      \item[(e)] We say $p \in \beta\bbN$ is an \textsl{idempotent
          (ultrafilter)} if  and only if $p + p = p$.
    \end{itemize}
  \end{defn}

The collection $\{\, \overline{A} : A \subseteq \bbN \,\}$ is a
basis for a topology on $\beta\bbN$.
We identify $\bbN$ with the principal ultrafilters by the function
$a \mapsto e(a)$. Then $\beta\bbN$ is simply the Stone-\v{C}ech
compactification of $\bbN$. 
Recall that $\beta\bbN$ is a compact Hausdorff space where $\bbN$ sits
densely inside.  

For all $p$, $q \in \beta\bbN$, the sum $p+q \in \beta\bbN$, and moreover,
$(\beta\bbN, +)$ is a semigroup where the addition on $\beta\bbN$ is
an extension of the usual addition on $\bbN$. 
However it is somewhat disconcerting to learn that since $(\beta\bbN,
+)$ contains a copy of the free group on $2^{2^\aleph_0}$ generators
\cite[Corollary 7.36]{Hindman:1998fk}, $(\beta\bbN,+)$ is a ``highly''
noncommutative semigroup.


We summarize our discussion by saying that $(\beta\bbN,
+)$ is a compact Hausdorff right-topological semigroup. 
The right-topological semigroup part means that for all $p \in
\beta\bbN$, the function $q \mapsto q+p$ is continuous.
It's a fact due to Robert Ellis \cite[Corollary 2.10]{Ellis:1969zr}
that every compact Hausdorff right-topological semigroup contains
idempotents. 
In the case of $\beta\bbN$, we have 
$2^{2^{\aleph_0}}$) idempotents \cite[Theorem
6.44]{Hindman:1998fk}. 

To demonstrate how all of this elaborate equipment can be put to use,
we give an easy proof of Schur's Theorem using an idempotent.

  \begin{schur}
    For every natural number $r$, if  \/ $\bbN = \bigcup_{i=1}^r C_i$,
    then there exist $i \in \{1, 2, \ldots, r\}$ and natural numbers
    $x$ and $y$ such that $\{\, x, y, x+y \,\} \subseteq C_i$.
  \end{schur}
  \begin{proof}
    Pick an idempotent ultrafilter $p + p = p \in \beta\bbN$. 
    Since $\bbN \in p$, we can, by Definition \ref{defn:uf}(4),
    that we can pick $i \in \{1, 2, \ldots, r\}$ such that $C_i \in p
    = p + p$. 
    Then by Definition \ref{defn:alg}(d) we have that $\{\, x \in \bbN : -x
    + C_i \in p \,\} \in p$.
    Definition \ref{defn:uf}(2) shows us that $C_i \cap
    \{\, x\in \bbN : -x + C_i \in p \,\} \in p$.
    Pick $x \in C_i$ such that $-x+C_i \in p$.
    Again, by Definition \ref{defn:uf}(2), $C_i \cap (-x+C_i) \in p$.
    Pick $y \in C_i \cap (-x+C_i)$.
    Then $\{\, x, y, x+y\,\} \subseteq C_i$.
  \end{proof}
It's possible to strengthen the conclusion of this theorem to show
that the set $\bigl\{\, \{x, y\} \subseteq \bbN : \{x, y, x+y\}
\subseteq C_i\,\bigr\}$ is infinite.
(This particular result can be derived easily by knowing that members
of nonprincipal ultrafilters are infinite, and idempotents in
$\beta\bbN$ are necessarily nonprincipal.)
Modifying the proof to produce Hindman's Theorem is mainly a matter of
notation and the right inductive argument.
Details can be found in \cite[Theorem 5.8]{Hindman:1998fk}. 

While the fact that $(\beta\bbN, +)$ has many idempotents, each giving
an easy proof of Hindman's Theorem, is interesting, there is another
class of structures in $(\beta\bbN, +)$ that sheds an interesting
light on Ramsey Theory. 
This class of structures are the ideals of $(\beta\bbN, +)$.
Before giving any formal definitions however, we turn our attention
back to combinatorial statements whose history runs almost parallel to
the Schur-type additive results we have seen so far.

\subsection{Szemer\'{e}di's Theorem}
Bartel Leendert van der Waerden, in \cite{Van-der-Waerden:1927fk},
proved that if the natural numbers are divided into finitely many
classes, then for every natural number $k$ at least one class contains
a $k$-term arithmetic progression. 
For expository convenience we will call a subset of the natural
numbers that contains a $k$-term arithmetic progression a
\textsl{$k$AP-set}.
In fact, using van der Waerden's Theorem and the pigeonhole principle
we can easily derive the superficially stronger statement that one
class is a $k$AP-set for every natural number $k$.
Almost a decade after van der Waerden published his proof, Erd\H{o}s
and Tur\'{a}n stated a conjecture in \cite{Erdos:1936fk} about a
generalization of van der Waerden's Theorem that continues to be the
source of vibrant research.

To state their conjecture we define the \textsl{upper (asymptotic)
  density} of $A \subseteq \bbN$ by 
  \[
    \overline{d}(A) = \limsup_{N\to\infty} \frac{|A \cap \{1, 2,
      \ldots, N\}|}{N}.
  \]
  \begin{sz}[Erd\H{o}s-Tur\'{a}n Conjecture]
    \label{sz:upperdensity}
    If $A \subseteq \bbN$ and $\overline{d}(A) > 0$, then $A$ is a
    $k$AP-set for every $k \in \bbN$.
  \end{sz}

The Erd\H{o}s-Tur\'{a}n Conjecture turned out to be both correct and a
hard problem; it was proved almost forty years later by Szemer\'{e}di in
\cite{Szemeredi:1975uq}. 
(If a set with positive upper density is divided into finitely many
classes, then at least one class has positive upper density.
Combining this fact with Szemer\'{e}di's Theorem gives us van der
Waerden's Theorem.)
Szemer\'{e}di's proof of his theorem uses an elementary, but
complicated, combinatorial argument. 
Shortly after this original proof, Furstenburg produced another proof
in \cite{Furstenberg:1977fk} that used Ergodic Theory. 

Furstenburg, for technical reasons, proved a slightly different form
of Szemer\'{e}di's Theorem. 
We choose to give this alternate form because their slight differences
will produce interesting algebraic consequences latter.
Define the \textsl{Banach density} of $A \subseteq \bbN$ by
  \begin{align*}
    d^*(A) &= \sup\bigl\{\, \alpha \in [0,1] : \hbox{for all $k \in
      \bbN$, there exist $n \ge k$ and $a \in \bbN$} \\
    &\hspace{5em}\hbox{such that $|A \cap \{\,a, a+1, \ldots,
      a+n-1\,\}| \ge \alpha\cdot n$} \,\bigr\}.
  \end{align*}
  \begin{sz}[Alternate form]
    \label{sz:banach}
    If $A \subseteq \bbN$ and $d^*(A) > 0$, then $A$ is a $k$AP-set
    for every $k \in \bbN$.
  \end{sz}
It's a straightforward exercise to show that if $\overline{d}(A) > 0$,
then $d^*(A) > 0$.
Therefore given Theorem \ref{sz:banach} we can derive Theorem
\ref{sz:upperdensity}.
The fact that Theorem \ref{sz:upperdensity} can be derive Theorem
\ref{sz:banach} was proved by Hindman in \cite[Theorem
2.1]{Hindman:1982fk}.
Hence we are justified in calling both versions Szemer\'{e}di's
Theorem. 

After these two initial proofs, a number of different types of proofs
have been discovered. 
The two latest proofs have been produced by Ben Green and Terence Tao
in \cite{Green:2010uq} and the (online) polymath group in
\cite{Polymath:2010kx}. 
However, despite the great variety of proofs produced so far, their is
currently no known ``algebraic proof'' of Szemer\'{e}di's Theorem.
By an algebraic proof I mean a proof that uses nothing but the
algebraic structure of $(\beta\bbN, +)$. 

Of course it is not obvious that such a proof, even in principle,
exists.
But, in the next section we give two pieces of evidence that an
algebraic proof may exist. 

\subsection{Known Algebraic Information on  Szemer\'{e}di's Theorem}
Given subsets $A$ and $B$ of $\beta\bbN$, we put $A+B = \{\, p + q : p
\in A \hbox{ and } q \in B \,\}$.
  \begin{defn}
    Let $\emptyset \ne I \subseteq \beta\bbN$.
      \begin{itemize}
        \item[(a)] $I$ is a \textsl{left ideal} if and only if
          $\beta\bbN + I \subseteq I$.
        \item[(b)] $I$ is a \textsl{right ideal} if and only if
          $I+\beta\bbN \subseteq I$.
        \item[(c)] $I$ is a \textsl{(two-sided) ideal} if and only if
          $I$ is both a left and right ideal.
      \end{itemize}
  \end{defn}

  \begin{defn}
    \begin{itemize}
      \item[(a)] $\mathcal{AP} = \{\, p \in \beta\bbN : \hbox{for all
          $A \in p$ $A$ contains a $k$-term arithmetic progression
          for every $k \in \bbN$} \,\}$.
      \item[(b)] $\Delta = \{\, p \in \beta\bbN : \hbox{for all $A \in
          p$, $\overline{d}(A) > 0$} \,\}$.
      \item[(c)] $\Delta^* = \{\, p \in \beta\bbN : \hbox{for all $A
          \in p$, $d^*(A) > 0$} \,\}$.
      \item[(d)] $\Delta_1 = \{\, p \in \beta\bbN : \hbox{for all $A
            \in p$, there exists $m \in \bbN$ such that
            $d^*(\bigcup_{n=1}^m -n+A) = 1$} \,\}$.
    \end{itemize}
  \end{defn}


\subsection{Algebraic Proof of Roth's Theorem}

  \begin{defn}
    For all $A \subseteq \bbN$, define the \textsl{upper asymptotic
      density} of $A$ by 
      \[
        \overline{d}(A) = \limsup_{N\to\infty} \frac{A \cap \{1, 2,
          \ldots, N\}}{N}.
      \]
  \end{defn}

Erd\H{o}s and Turan conjectured that if $\overline{d}(A) > 0$, then
$A$ contains a $k$-term arithmetic progression for every natural
number $k$.
Their conjecture turned out to be correct and was proved almost forty
years later by Szemer\'{e}di in \cite{Szemeredi:1975uq}. 
To state Szemer\'{e}di's Theorem in algebraic language we introduce
the following two definitions.
  \begin{defn}
    \begin{itemize}
      \item[(a)] $\Delta = \{\, p \in \beta\bbN : \overline{d}(A) > 0
        \hbox{ for all } A \in p \,\}$.
      \item[(b)] $\mathcal{AP} = \{\, p \in \in \beta\bbN : \hbox{for
          all $A \in p$ } A \hbox{ contains a $k$-term arithmetic
          progression for every $k \in \bbN$} \,\}$.
    \end{itemize}
  \end{defn}

Two important results in Ramsey Theory are Szem\'{e}rdi's Theorem and
Roth's Theorem. 
To state these theorems we introduce the familiar notion of \textsl{upper
asymptotic density}
  \begin{defn}
    Given $A \subseteq \bbN$, define $\overline{d}(A) = \limsup_{N \to
      \infty} (A \cap \{1, 2, \ldots, N\})/ N$.
  \end{defn}
Szem\'{e}rdi's Theorem states that if $\overline{d}(A) > 0$, then $A$
contains arbitrarily long arithmetic progressions. 
This particular theorem is hard to prove and has a long history. 
Van der Waerden proved $\la\hbox{blah}\ra$, then Erd\H{o}s and Turan
was interested in which ``size'' of guarantees the conclusion to van
der Waerden's Theorem, they correctly conjectured that if
$\overline{d}(A) >0$, then $A$ contains arbitrarily long arithmetic
progressions. 
The proof of this conjecture was not immediate put proceeded in
steps. 
The first big step was provided by Roth in \cite{Roth:1953fk}. 
In this paper he proved the following $\la\hbox{Roth's Theorem}\ra$.
Next Szem\'{e}rdi gave a combinatorial proof of $\la\hbox{Roth's
  Theorem}\ra$ and extended it to a 4-term arithmetic progression. 
Finally Szem\'{e}rdi was able to prove the full case in
\textsc{SzPaper}. 
Soon after another proof of Szem\'{e}rd's Theorem was provided by
Fursternburg in \textsc{ERTpaper} that used Ergodic Theory. 
Since then several more proofs of Szem\'{e}di's Theorem have appeared
\textsc{cite all of the various version included DHJ}. 
It's an empirical fact that most Ramsey Theory statements are either
about idempotents or ideals.
For instance, van \dots


\bibliographystyle{plain}
\bibliography{references}
\end{document}

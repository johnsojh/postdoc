\documentclass[12pt]{article}

\usepackage[doublespacing]{setspace}
\usepackage{amsthm, amssymb, amsmath}
\usepackage[margin=1in]{geometry}
\usepackage{endnotes}

\newtheoremstyle{plain}{3mm}{3mm}{\slshape}{}{\bfseries}{.}{.5em}{}
\theoremstyle{plain}
\newtheorem{thm}{Theorem}[section]
\newtheorem{vdW}[thm]{Van der Waerden's Theorem}
\newtheorem{FS}[thm]{Hindman's Theorem}
\newtheorem{MBR}[thm]{Multiple Birkhoff Recurrence Theorem}
\newtheorem{recur}[thm]{Recurrence Theorem}
\newtheorem{OCST}[thm]{ Central Sets Theorem}
\newtheorem{cor}[thm]{Corollary}
\newtheorem{prop}[thm]{Proposition}
\newtheorem{lem}[thm]{Lemma}
\newtheorem{claim}[thm]{Claim}
\newtheorem{ques}[thm]{Question}
\newtheorem{conj}[thm]{Conjecture}
\newtheorem{fact}[thm]{Fact}

\theoremstyle{definition}
\newtheorem{defn}[thm]{Definition}
\newtheorem{example}[thm]{Example}
\newtheorem{rmk}[thm]{Remark}

\newcommand{\la}{\langle}
\newcommand{\ra}{\rangle}
\newcommand{\bbN}{\mathbb{N}}
\newcommand{\bbZ}{\mathbb{Z}}
\newcommand{\ds}{(X, \la T_s \ra_{s\in S})}
\newcommand{\setfunc}[2]{\hbox{${}^{\hbox{$#1$}}\hskip -2 pt #2$}}
\newcommand{\Pf}{\mathcal{P}_f}

\font\bigmath=cmsy10 scaled \magstep 3
\newcommand{\bigtimes}{\hbox{\bigmath \char'2}}

\newcommand{\cchi}{\raise 2 pt \hbox{$\chi$}}

\title{A Dynamical Characterization of $C$-sets}
\author{John H.~Johnson}
\date{}

\begin{document}
\maketitle
\section{Introduction}
Furstenburg in \cite[Chapter 8]{Furstenberg:1981fk}, using notions of
topological dynamics, defined the concept of a central subset of
$\bbN$.
He was able to prove an important result about central sets
called the Central Sets Theorem.

For a set $X$ we let $\Pf(X)$ denote the set of all finite nonempty
subsets of $X$.
Given sets $A$ and $B$, we let $\setfunc{A}{B}$ represent the set of
all function with domain $A$ and codomain $B$. 
With this notation we now give the Central Sets Theorem.
\begin{OCST}[{\cite[Theorem 8.21]{Furstenberg:1981fk}}]
  Put $\mathcal{T} =
  \setfunc{\bbN}{\bbZ}$, and let $C \subseteq \bbN$ be central.
  Let $F \in \mathcal{P}_f(\mathcal{T})$. 
  Then there exist sequences $\la a_n \ra_{n=1}^\infty$ in $S$ and
  $\la H_n \ra_{n=1}^\infty$ in $\mathcal{P}_f(\bbN)$ such that
    \begin{itemize}
      \item[(1)] $\max H_n < \min H_{n+1}$ for all $n \in \bbN$, and
      \item[(2)] for all $G \in \mathcal{P}_f(\bbN)$ and every $f
        \in F$, we have
        \[
          \sum_{n \in G}\Bigl(a_n + \sum_{t \in H_n} f(t)\Bigr) \in C.
        \]
    \end{itemize}
\end{OCST}

Central sets can be defined for all semigroups; and, central sets
admit a simple definition using the algebraic structure of the
Stone-\v{C}ech compactification of a discrete semigroup. 
The equivalence between the dynamical definition of a central set and
its algebraic definition can be found in \cite[Chapter
19]{Hindman:1998fk}.
We now give a brief review of the algebraic structure of the
Stone-\v{C}ech compactification of a discrete semigroup.

Given a discrete semigroup $(S, \cdot)$, we let $\beta S$ be the
collection of all ultrafilters on $S$, and identify the principal
ultrafilters with the points of $S$. 
For all $A \subseteq S$, we let $\overline{A} = \{\, p \in \beta S : A
\in p\,\}$. 
Then $\{\, \overline{A} : A \subseteq S \,\}$ is a basis for a compact
Hausdorff topology on $\beta S$. 
We can extend the semigroup operation on $S$ to $\beta S$ in the
following manner.
For $p$, $q \in \beta S$ and for all $A \subseteq S$, $A \in p \cdot
q$ if and only if $\{\, x \in S: x^{-1}A \in q \,\} \in p$, where
$x^{-1}A = \{\, y \in S : xy \in A \,\}$. 

With our definitions, $(\beta S, \cdot)$ becomes a compact Hausdorff
right-topological semigroup with $S$ contained in the topological
center of $\beta S$.
What makes $\beta S$ a right-topological semigroup is that for all $q
\in \beta S$, the map $\rho_q : \beta S \to \beta S$ given by
$\rho_q(p) = pq$ is continuous. 
A point $x \in \beta S$ is in the topological center of $\beta S$ if
and only if the map $\lambda_x : \beta S \to \beta S$, defined by
$\lambda_x(p) = xp$ is continuous.

Given $\emptyset \ne J \subseteq \beta S$, we call $J$ a \textsl{left
  ideal} if $\beta S \cdot J \subseteq J$; we call $J$ a \textsl{right
  ideal} if $J \cdot \beta S \subseteq J$; and we call $J$ a
\textsl{(two-sided) ideal} if $J$ is both a left and right ideal.

It's a fact \cite[Theorem 2.8]{Hindman:1998fk} that any compact Hausdorff right-topological semigroup $T$ has
a smallest ideal, denoted by $K(T)$. 
We can now give the definition of a central subset of a discrete
semigroup $S$.

\begin{defn}
  Let $(S, \cdot)$ be a discrete semigroup, and let $C \subseteq S$.
  We say $C$ is a \textsl{central} set if and only if there exists
  $p \in K(\beta S)$ such that $p = p \cdot p$ and $C \in p$.
\end{defn}

We can now formulate the Central Sets Theorem for (commutative)
semigroups.
The reason we give the theorem only for commutative semigroups now,
is that the general version of the Central Sets Theorem is quite
complicated to state. 
(We will eventually give the ``full'' version of the Central Sets
Theorem however.)
\begin{thm}
    \label{thm:cst2}
    Let $(S,+)$ be a commutative semigroup, put $\mathcal{T} =
    \setfunc{\bbN}{S}$, and let $C \subseteq S$ be central.
    Let $F \in \mathcal{P}_f(\mathcal{T})$. 
    Then there exist sequences $\la a_n \ra_{n=1}^\infty$ in $S$ and
    $\la H_n \ra_{n=1}^\infty$ in $\mathcal{P}_f(\bbN)$ such that
      \begin{itemize}
        \item[(1)] $\max H_n < \min H_{n+1}$ for all $n \in \bbN$, and
        \item[(2)] for all $G \in \mathcal{P}_f(\bbN)$ and every $f
          \in F$, we have
          \[
            \sum_{n \in G}\Bigl(a_n + \sum_{t \in H_n} f(t)\Bigr) \in C.
          \]
      \end{itemize}
  \end{thm}

There are currently four different versions of the Central Sets
Theorem. 
Theorem \ref{thm:cst2} is the second version.
The most recent version of the Central Sets Theorem for
commutative semigroups is the following. 
 \begin{thm}[{\cite[Theorem 2.2]{De:2008uq}}]
    \label{thm:newcst}
    Let $(S,+)$ be a commutative semigroup, put $\mathcal{T} =
    \setfunc{\bbN}{S}$, and let $C \subseteq S$ be central. 
    Then there exist functions $\alpha : \mathcal{P}_f(\mathcal{T})
    \to S$ and $H : \mathcal{P}_f(\mathcal{T}) \to \mathcal{P}_f(\bbN)$ such
    that
      \begin{itemize}
        \item[(1)] if $F$, $G \in \mathcal{P}_f(\mathcal{T})$ and $F
          \subsetneq G$, then $\max H(F) < \min H(G)$, and
        \item[(2)] whenever $m \in \bbN$, $\{ G_1, G_2, \ldots, G_m
          \}$ is a finite sequence in $\mathcal{P}_f(\mathcal{T})$
          with $G_1 \subsetneq G_2 \subsetneq \cdots \subsetneq G_m$,
          and for each $i \in \{1, 2, \ldots, m\}$, $f_i \in G_i$ we
          have
          \[
            \sum_{i=1}^m\Bigl(\alpha(G_i)+\sum_{t \in H(G_i)}
            f_i(t)\Bigr) \in C.
          \]
      \end{itemize}
  \end{thm}
Given that the Central Sets Theorem guarantees that central sets have
strong combinatorial properties, it's natural to ask if the Central
Sets Theorem characterizes centrals sets. 
It doe not since there are many sets that satisfy the conclusion of
the Central Sets Theorem but are not central.
See \cite[Theorem 4.4]{Hindman:1996uq} for the first example of such a
set. 

Since most applications of central sets involves only the use of the
Central Sets Theorem, recently the focus has shifted to the study of
those sets which satisfy the conclusion of Theorem \ref{thm:newcst}. 
We will now give a definition of these sets, but we must warn the
reader that the definition is complicated. 
For the reader that doesn't want to be bothered with such
complications, all you need to know for the main result is that $C$-sets satisfy the
conclusion of the Central Sets Theorem and Theorem \ref{thm:csetid}
below. 
   \begin{defn}
    Let $(S, \cdot)$ be a semigroup.
      \begin{itemize}
        \item[(a)] For each $m \in \bbN$, put 
          \begin{align*}
            \mathcal{I}_m &= \Bigl\{\, \bigl( H(1), H(2), \ldots, H(m)
            \bigr) : \hbox{$H(i) \in \mathcal{P}_f(\bbN)$ for all $i
              \in \{1, 2, \ldots, m\}$} \\
            &\hbox{ and $\max H(i) < \min H(i+1)$
            for all $i \in \{1, 2, \ldots, m-1\}$} \Bigr\}
         \end{align*}
        \item[(b)] $\mathcal{T} = \setfunc{\bbN}{S}$. 
        \item[(c)] Given $m \in \bbN$, $a \in S^{m+1}$, $H \in
          \mathcal{I}_m$, and $f \in \mathcal{T}$, put 
            \[
            x(m, a, H, f) = \prod_{i=1}^m\Bigl(a(i)\prod_{t \in H(i)}
            f(t) \Bigr)a(m+1).
            \]
        \item[(d)] A subset $A \subseteq S$ is a \textsl{$J$-set} if
          and only if for all $F \in \mathcal{P}_f(\mathcal{T})$ there
          exist $m \in \bbN$, $a \in S^{m+1}$, and $H \in \mathcal{I}_m$
          such that for all $f \in F$ we have $x(m,a,H,f) \in A$.
        \item[(e)] $J(S) = \{\, p \in \beta S : \hbox{$A$ is a $J$-set
            for all $A \in p$} \,\}$.
      
        \item[(f)] A subset $C \subseteq S$ is a \textsl{$C$-set} if
          and only if there exist functions $m :
          \mathcal{P}_f(\mathcal{T}) \to \bbN$, $\alpha \in
          \bigtimes_{F \in \mathcal{P}_f(\mathcal{T})} S^{m(F)+1}$,
          and $H \in \bigtimes_{F \in \mathcal{P}_f(\mathcal{T})}
          \mathcal{I}_{m(F)}$ such that

          \begin{itemize}
            \item[(1)] if $F$, $G \in \mathcal{P}_f(\mathcal{T})$ and $F
              \subsetneq G$, then $\max H(F)\bigl(m(F)\bigr) < \min
              H(G)(1)$, and
           \item[(2)] whenever $n \in \bbN$, $\{ G_1, G_2, \ldots, G_n
              \}$ is a finite sequence in $\mathcal{P}_f(\mathcal{T})$
              with $G_1 \subsetneq G_2 \subsetneq \cdots \subsetneq G_n$,
              and for each $i \in \{1, 2, \ldots, n\}$, $f_i \in G_i$ we
              have
              \[
              \prod_{i=1}^n\Bigl(x\bigl(m(G_i), \alpha(G_i), H(G_i),
              f_i\bigr)\Bigr) \in C
             \]
      \end{itemize}
    \end{itemize}
  \end{defn}

  \begin{thm}
    \label{thm:csetid}
    Let $C \subseteq S$. 
    Then $C$ is a $C$-set if and only if there exists $p\cdot p = p
    \in J(S)$ such that $C \in p$.
  \end{thm}
  \begin{proof}
    This is \cite[Theorem 3.8]{De:2008uq}.%
  \end{proof}

Since central sets where originally defined using notions of
topological dynamics, it is natural to ask whether $C$-sets
admit a dynamical characterization.
Our main result \ref{thm:dyncsets}, provides such a characterization
using the methods in \cite{Burns:2007uq}.
Before we can begin we will need to define what we mean by a dynamical
system, and provide a connection to $\beta S$.


\section{Dynamical Systems and $\beta S$}
\begin{defn}
    A pair $\ds$ is a \textsl{dynamical system} if and only if
      \begin{itemize}
        \item[(1)] $X$ is a compact Hausdorff space;
        \item[(2)] $S$ is a semigroup;
        \item[(3)] $T_s : X \to X$ is continuous for all $s \in S$;
          and
        \item[(4)] $T_s \circ T_t = T_{st}$ for all $s$, $t \in S$.%
       \end{itemize}
  \end{defn}
 Let $\ds$ be a dynamical system and define $T : S \to
    \setfunc{X}{X}$ by $T(s) = T_s$.
Giving $\setfunc{X}{X}$ the product topology and, as usual, taking $S$ to be 
discrete, we have that the function $T :
S \to \setfunc{X}{X}$ is a continuous semigroup
homomorphism into a compact space.
Therefore by \cite[Theorem 3.27]{Hindman:1998fk} we can produce a
continuous extension $\widetilde{T}$ of $T$.
(More directly, $\widetilde{T}$ is defined for each $p \in \beta S$ by
$\widetilde{T}(p) \in \bigcap \{\, c\ell\bigl( T[A] \bigr) : A \in p \,\}$, where
the closure is in $\setfunc{X}{X}$.)
Now by \cite[Theorem 2.22(a)]{Hindman:1998fk} the space $\setfunc{X}{X}$ is
a compact right-topological semigroup.
In order to show that $\widetilde{T}$ is a semigroup homomorphism it
suffices by, \cite[Corollary 4.22]{Hindman:1998fk}, to show that for all
$s \in S$, the
map $\lambda_{T(s)}$ is continuous.
However by \cite[Theorem 2.2(b)]{Hindman:1998fk} we have that
$\lambda_{T(s)}$ is continuous if and only if $T(s)$ is continuous. 
Since $\ds$ is a dynamical system and $T(s) = T_s$ we know by
definition that $T(s)$ is continuous. 
Hence $\widetilde{T} : \beta S \to \setfunc{X}{X}$ is a continuous semigroup
homomorphism.

By using the map $\widetilde{T} : \beta S \to X$, we can define
    $T_p : X \to X$, for $p \in \beta S$, as $T_p =
    \widetilde{T}(p)$. 
    Since $\widetilde{T}$ is a semigroup homomorphism we immediately
    conclude that $T_p \circ T_q = T_{pq}$ for all $p$, $q \in \beta
    S$.

    \begin{rmk}
      It's important to note that in general $(\beta S, \la T_p \ra_{p
  \in \beta S})$ is not a dynamical system \cite[Theorem 6.10 and
Remark 6.11]{Hindman:1998fk}.
    \end{rmk}

This remark is somewhat disappointing since we lose the ``dynamical
part'' when extending the dynamical system to $\beta S$.
However even with this lost we will be able to prove something
intelligible about certain dynamical systems. 
Intuitively, the points of $S^*$ can be thought of as points at
infinity. 
Since in dynamical systems we are often concerned with the
``long-run'' behavior of our maps $T_s$ we will often be able to
correspond any interesting long run behaviors with a point at infinity
in $S^*$.

To make this idea precise, we will be using the notion of a limit
along an ultrafilter. 

  \begin{defn}
    \label{defn:plim}
    Let $S$ be a discrete space, $p \in \beta S$, $X$ a compact
    Hausdorff topological space, $\la x_s : s \in S \ra$ a family
    of points in $X$, and $y \in X$.
    Then \hbox{$p$-$\displaystyle\lim_{s \in S} x_s = y$} if and only
    if for every
    neighborhood $U$ of $y$ we have $\{\, s \in S : x_s \in U \,\} \in p$.
  \end{defn}
\begin{rmk}
    Our definition of a \hbox{$p$-limit} is a bit more restrictive
    then usual.
    Normally the definition only takes $X$ be any topological space.
    We placed these extra restrictions on $X$ to ensure that every
    \hbox{$p$-limit} exists and is unique.
  \end{rmk}
\begin{prop}
    \label{prop:dsplim}
    Let $\ds$ be a dynamical system.
    Then for every $p \in \beta S$ and each $x \in X$ we have $T_p(x)
    = \hbox{$p$--$\lim_{s \in S} T_s(x)$}$.
  \end{prop}
  \begin{proof}
    By definition, we need to show that for all neighborhoods $U$ of
    $T_p(x)$, we have $\{\, s \in S : T_s(x) \in U \,\} \in p$. 
    Define $\pi_x : \setfunc{X}{X} \to X$ by $\pi_x(f) = f(x)$ and let
    $U$ be a neighborhood of $T_p(x)$.
    Note that $\pi_x^{-1}[U]$ is a neighborhood of $T_p$.
    By definition of $T_p$, we have that for all $A \in p$,
    $\pi_x^{-1}[U] \cap \{\, T_s : s \in A \,\} \ne \emptyset$. 
    Hence for all $A \in p$, there exists $s \in A$ such that $T_s(x)
    \in U$. 

    Suppose there exists a neighborhood $U$ of $T_p(x)$ such that
    $\{\, s \in S : T_s(x) \in U \,\} \not\in p$. 
    Put $A = \{\, s \in S : T_s(x) \not\in U\,\}$.
    Then $A \in p$.
    Therefore there exists $s \in A$ such that $T_s(x) \in U$ (by our
    first paragraph) and $T_s(x) \not\in U$ (by our definition of
    $A$), a contradiction.
  \end{proof}

With these preliminaries out of the way, we can now provide a
dynamical characterization of $C$-sets.

\section{Dynamical Characterization of $C$-sets}
\begin{defn}
    \label{defn:JSUR}
    Let $\ds$ be a dynamical system, and let $x$, $y \in X$. 
    The pair $(x,y)$ is \textsl{jointly sparsely uniformly recurrent}
    (we'll abbreviate this to JSUR) if and only if $\{\, s \in S :
    \hbox{$T_s(x) \in U$ and $T_s(y) \in U$} \,\}$ is a $J$-set for every
    neighborhood $U$ of $y$.%
  \end{defn}


  \begin{lem}
    \label{lem:JSUR}
    Let $\ds$ be a dynamical system, and let $x$, $y \in X$.
    The following statements are equivalent.
    \begin{itemize}
      \item[(a)] The pair $(x, y)$ is JSUR.
      \item[(b)] There exists $r \in J(S)$ such that $T_r(x) = y = T_r(y)$.
      \item[(c)] There exists an idempotent $r \in J(S)$ such that $T_r(x)
        = y = T_r(y)$. 
    \end{itemize}
  \end{lem}
  \begin{proof}
    (a) $\Rightarrow$ (b). 
    For each neighborhood $U$ of $y$, put 
      \[  
        B_U = \{\, s \in S : \hbox{$T_s(x) \in U$ and $T_s(y) \in U$}
        \,\}.
      \]
    By assumption each $B_U$ is a $J$-set. 
    We now show that the collection $\{\, B_U : \hbox{$U$ is a
      neighborhood of $y$} \,\}$ is closed under finite intersection
    by showing that, for all neighborhoods $U$ and $V$ of $y$ we have $B_{U
      \cap V} = B_U \cap B_V$.
    Let $s \in S$, then 
      \begin{align*}
        s \in B_{U \cap V} &\iff \hbox{$T_s(x) \in U \cap V$ and $T_s(y) \in U
        \cap V$}, \\
      &\iff \hbox{$T_s(x) \in U$, $T_s(x) \in V$, $T_s(y) \in U$, and
        $T_s(y) \in V$}, \\
      &\iff s \in B_U \cap B_V.
      \end{align*}
    
    By \cite[Theorem 2.14]{Hindman:2010fk}, we know that every $J$-set
    of $S$ is partition regular. 
    Therefore by \cite[Theorem 3.11 (b)]{Hindman:1998fk},  we can pick 
    $r \in J(S)$ such that $\{\, B_U : \hbox{$U$ is a neighborhood of
      $y$}\,\} \subseteq r$. 
    
    Now for all neighborhoods $U$ of $y$, we have $B_U \subseteq \{\,
    s \in S : T_s(x) \in U \,\}$ and $B_U \subseteq \{\, s \in S :
    T_s(y) \in U \,\}$. 
    Therefore $\{\, s \in S : T_s(x) \in U\,\} \in r$ and $\{\, s \in
    S : T_s(y) \in U \,\} \in r$. 
    By Definition \ref{defn:plim}, we can conclude that
    $r$-$\displaystyle\lim_{s\in S} T_s(x) = y$ and
    $r$-$\displaystyle\lim_{s \in S} T_s(y) = y$. 
    Hence by Proposition \ref{prop:dsplim}, we have $T_r(x) =
    r$-$\displaystyle\lim_{s \in S} T_s(x) = y = r$-$\displaystyle\lim_{s \in S}
    T_s(y) = T_r(y)$.
  
    (b) $\Rightarrow$ (c).
    Put $M  = \{\, r \in J(S) : T_r(x) = y = T_r(y) \,\}$. 
    We'll show that $M$ is a nonempty compact subsemigroup of $J(S)$. 
    If we can show this, then we can pick an idempotent in $M$ and our
    result follows.
    The fact that $M \ne \emptyset$ follows from our assumption. 
    To see that $M$ is compact it suffices to show that $M$ is
    closed. 
    Let $r \not\in M$, then either $T_r(x) \ne y$ or $T_r(y) \ne y$. 
    First, assume that $T_r(x) \ne y$. 
    By Definition \ref{defn:plim} and Proposition \ref{prop:dsplim},
    pick a
    neighborhood $U$ of $y$ such that 
    $\{\, s \in S : T_s(x) \in U \,\} \not\in r$.
    Put $A = \{\, s \in S : T_s(x) \in U \,\}$ and note that $S
    \setminus A \in r$ and $\overline{S \setminus A} \cap M =
    \emptyset$.
    (If $p \in \overline{S \setminus A} \cap M$, then $A = \{\, s \in
    S : T_s(x) \in U \,\} \in p$ by Definition \ref{defn:plim} and
    Proposition \ref{prop:dsplim}.
    We have also have that $S \setminus A \in p$.
    Hence $\emptyset = A \cap (S \setminus A) \in p$, a
    contradiction.)
    Now assume that $T_r(y) \ne y$. 
    By Definition \ref{defn:plim} and \ref{prop:dsplim}, pick a
    neighborhood $U$ of $y$ such that
    $\{\, s \in S : T_s(y) \in U \,\} \not\in r$.
    Put $A = \{\, s \in S : T_s(y) \in U \,\}$ and note that $S
    \setminus A \in r$ and $\overline{S \setminus A} \cap M =
    \emptyset$.
    Hence $M$ is a nonempty closed subset of $J(S)$.

    To see that $M$ is a subsemigroup, let $q$, $r \in M$.
    Then $T_{qr}(x) = T_q\bigl(T_r(x)\bigr) = T_q(y) =
    T_q\bigl(T_r(y)\bigr) = T_{qr}(y)$ and $T_q(y) = y$. 
    Hence $qr \in M$. 
    
    (c) $\Rightarrow$ (a).
    Pick $r$ as guaranteed in (c). 
    Let $U$ be a neighborhood of $y$. 
    Then $\{\, s \in S : T_s(x) \in U \,\} \in r$ and $\{\,  s \in S :
    T_s(y) \in U \,\} \in r$.
    Hence $\{\, s \in S : \hbox{$T_s(x) \in U$ and $T_s(y) \in U$}
    \,\} \in r$.  
  \end{proof}

  \begin{thm}
    \label{thm:dyncsets}
    Let $(S,\cdot)$ be a semigroup and $A \subseteq S$. 
    Then $A$ is a $C$-set if and only if there exist a dynamical
    system $\ds$ with points $x$, $y \in X$ where $(x,y)$ is JSUR, and
    a neighborhood $U$ of $y$ such that $A = \{\, s \in S : T_s(x) \in
    U \,\}$.
  \end{thm}
  \begin{proof}
    ($\Rightarrow$) Let $A \subseteq S$ be our $C$-set, and by
    Theorem \ref{thm:csetid} pick an idempotent $r \in J(S)$ such that
    $A \in r$. 
    Let $R = S \cup \{e\}$ be the semigroup with an identity $e$
    adjoined to S. 
    (For expository convenience, we still add this new identity even
    if $S$ already contains an identity.)
    Give $\{0,1\}$ the discrete topology, and take
    $\setfunc{R}{\{0,1\}}$ to have
    the product topology, and put $X = \setfunc{R}{\{0,1\}}$.
    Hence $X$ is a compact Hausdorff space.
    For each $s \in S$, define $T_s : X \to X$ by $T_s(f) = f \circ
    \rho_s$. 
    % Perhaps I should sketch the argument here?
    By \cite[Theorem 19.14]{Hindman:1998fk}, $\ds$ is a dynamical
    system. 
    
    Now let $x = \cchi_A$ be the characteristic function of $A$, and
    put $y = T_r(x)$.
    % Perhaps I should provide a proof since there is not one
    % explicitly given in the book?
    Then by \cite[Remark 19.13]{Hindman:1998fk}, we have that $T_r(y)
    = T_r\bigl(T_r(x)\bigr) = T_{rr}(x) = T_r(x) = y$.
    Therefore by (c) in Lemma \ref{lem:JSUR}, the pair $(x, y)$ is JSUR.

    Put $U = \{\, w \in X : w(e) = y(e) \,\}$, and note that $U =
    \pi^{-1}\bigl[\{y(e)\}\bigr]$ and so $y \in U$. 
    Hence $U$ is a (subbasic) open neighborhood of $y$.
    To help us show that $U$ is the neighborhood of $y$ we are looking
    for we will show that $y(e) = 1$.
    Since $y = T_r(x)$ we have that $\{\, s \in S : T_s(x) \in U \,\}
    \in r$ by Definition \ref{defn:plim} and Proposition \ref{prop:dsplim}.
    Since $A \in r$, we can pick $s \in A$ such that $T_s(x) \in U$. 
    Then by definition of $U$ and our choice of $T_s(x) \in U$, we
    have $y(e) = T_s(x)(e) = x\bigl(\rho_s(e)\bigr) = x(es) = x(s) =
    \chi_{A}(s) = 1$. 
    Finally, given $s \in S$, we have
      \begin{align*}
        s \in A &\iff \chi_{A}(s) = 1, \\
                &\iff x(s) = 1, \\
                &\iff x(es) = 1, \\
                &\iff (x \circ \rho_s)(e) = 1, \\
                &\iff T_s(x)(e) = 1 = y(e), \\
                &\iff T_s(x) \in U.
      \end{align*}
   Hence $A = \{\, s \in S : T_s(x) \in U \,\}$. 
   
   ($\Leftarrow$) Let $\ds$ and let the points $x$, $y \in X$ be given as
   guaranteed.
   By Theorem \ref{thm:csetid}, pick an idempotent $r \in J(S)$
   such that $T_r(x) = y = T_r(y)$. 
   Since $U$ is a neighborhood of $y$ and $T_r(x) = y$, we have that
   $A \in r$ by Definition \ref{defn:plim} and Proposition \ref{prop:dsplim}. 
 \end{proof}

\bibliographystyle{amsplain}
\bibliography{references}

\end{document}

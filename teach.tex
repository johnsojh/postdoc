\documentclass[12pt]{article}

\usepackage[margin=1in]{geometry}
\usepackage[doublespacing]{setspace}

\title{Teaching Statement}
\author{John H.~Johnson}

\begin{document}
\maketitle
I believe that, fundamentally, mathematics can be viewed as the
synthesis of three components.
The logical underpinning of a particular mathematical situation gives
``the rules of the game''.
Another component is the intuitive content of mathematics.
At a basic level, mathematics codify \textsl{some} of our intuition on
numbers and various abstract objects.
Finally, the varied applicability of mathematics is the third major
component of the subject.
When teaching mathematics, all three of these components and their
multifaceted inter-relationships should always be emphasized to the
student. 
To give a small example of the effectiveness of this approach let me
describe a mathematical activity I performed with high school
sophomore during the Fall of 2008. 

Since the activity occurred around the time of the United States
Presidential Election, I wanted the students to observe how voting
rules can effect the outcomes of fair elections.
To illustrate this concept, we each voted on our personal
most-valuable-player (MVP) in basketball for that year.
Using our voting results, we applied three different voting rules to
produce three different MVP winners.
Since this result was surprising to the students we decided to
investigate further. 
(I was surprised during the ``testing phase'' of the activity.)
After running the MVP voting experiment several more times, we
observed that this ``voting paradox'' happens frequently. 
This lead to several interesting conjectures, and a more detailed
analysis on how the concept of the ``will of the voters'' is not as
clean-cut as commonly believed.

Through this activity the students where exposed to the three
components of mathematics.
At first, the students had certain intuitive expectations on how
voting rules effect the outcome of elections.
Like everyone their initial intuitions ranged from correct, to
incorrect, to vaguely articulated.
The need to modify our intuition lead us to look into the voting rules
more logically.
Finally, the applicability of our investigation is obvious since it is
important to know how various voting rules can affect the will of the
voters. 

By emphasizing these three aspects, the students were better able to
internalize, appreciate, enjoy, and connect with mathematics. 
Hence, I believe that my main role as a teacher is to communicate and
facilitate students' understanding of these three concepts.

\end{document}
